% !TeX spellcheck = en_GB
% !TEX program = xelatex+makeindex+bibtex
\documentclass[final,a4paper,11pt]{article}
\input{../library/preamble.tex}
\input{../library/style.tex}
\addbibresource{../library/bibliography.bib}
\author{Erwin de Haan}
\title{User testing results for Ecoptimizer}
\date{\today}
\newlength{\resulttablecolone}
\setlength{\resulttablecolone}{4 cm}
\newlength{\resulttablecoltwo}
\setlength{\resulttablecoltwo}{11.3 cm}
\begin{document}
\maketitle
For this user testing, I did not choose to make categories based on user age but on computer literacy. I made three groups.
\begin{enumerate}
\item Beginner
\item Advanced
\item Expert
\end{enumerate}
I let people choose which category they think they belong in. This is probably affected by the amount of self esteem, but it could also be seem as a good thing, people with the same level of confidence in using a computer test almost the same way. People with more confidence are more adventurous and thus more likely to use a new thing and willing to understand it.
\section*{Tests}
I set up the first-wave test protocol in the following way.
\begin{enumerate}
\item Explain to the user what I want them to do. In this case select the mode of transport they are most comfortable to use to travel to Amsterdam to visit the city. (All test participants are at least 30 kilometers from Amsterdam.)
\item Give the user a computer or tablet with the application open. Use a device the user is familiar with, almost everything has a web browser.
\item Record the actions taken and remarks made during testing. If people give up, explain that the application is essentially a form of Google Maps, this should give people enough confidence to work with the application.
\item \textbf{Assessment of functionality} Once a route is found, ask the user why they picked the chosen mode of transport and why not the other options. And if they feel the functionality adds enough to the map experience to warrant use over other map applications. (important)
\item \textbf{Assessment of look-and-feel} Have the user give any other remark on the design, look-and-feel and other aspects of the application.
\end{enumerate}
The later waves are kind of setup to get a bit of iterative design going, mainly following the same protocol.
\section*{Results, Iterative testing \& reflection}
Some reactions are paraphrased due to translation from Dutch to English. I took a little liberty from the protocol some times, especially for the repeated tests. I put after every test item a little reflection to put the test in context and to give a little extra info.

\subimport{waves/}{first-wave}

\subimport{waves/}{second-wave}

\subimport{waves/}{third-wave}

\subimport{waves/}{fourth-wave}

\section*{Final conclusions}
The design underwent pretty big changes from nearly unusable in wave one to near-final in wave four. The way I did the user testing really helped in validating the changes (I never told people what was different, only on a direct question). I would have loved to have many hundred of people stress test the design, but this did not work out. The reddit community dedicated to such things quickly rejected it because it was a web application and that is not a true application in the traditional sense. Got some feedback though like putting in the source of the data.


\end{document}