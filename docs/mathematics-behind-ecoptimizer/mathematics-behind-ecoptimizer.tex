% !TeX spellcheck = en_GB
% !TEX program = xelatex+makeindex+bibtex
\documentclass[final,a4paper,11pt]{article}
\input{../library/preamble.tex}
\input{../library/style.tex}
\addbibresource{../library/bibliography.bib}
\author{Erwin de Haan}
\title{The Mathematics Behind Ecoptimizer}
\date{\today}
\begin{document}
\maketitle
The Ecoptimizer is a tool to see how much impact your travel has on the world.
\section{Impact}
I have a table of values with points per kilometer. Google gives back each route in steps with the type of travel attatched (Table \ref{tab:google-transport-modes}).
The main calculation is in essence a small LCA-style product and sum.
Each kilometer of each different mode of transport is summed and then multiplied with the pt/km value for that mode.
A total is calculated and an overall pt/km value to give a good way to compare journeys of differing lengths.

The values are heavily optimised for the Netherlands (ex. stop density for public transport and urban density for car usage), but should work fairly well in differing countries. The base data is for the UK. The ones that I have no data for are very rare in the Netherlands and most of the time a very short leg on the journey. So their impact is near insignificant.

I personally have a problem with the car figure. It would need to be differentiated between city driving and highway driving. This is currently not possible using the Google Maps API. So this average is based on the urban density of the UK. It's the value for a small family car with average occupation (1.6 persons).
\begin{table}[H]
\caption{Google Transport Modes and their points per kilometer \cite{ben-lane-lca}}
\label{tab:google-transport-modes}
\begin{tabular}{l|r|p{4cm}|p{4cm}}
\textbf{Type} & \textbf{pt/km} & \textbf{Description} & \textbf{ID \& remarks} \\
\hline
\hline
DRIVING & 30.30 & Car & ID 201 \\
WALKING & 5.41 & Walking & ID 1 \\
CYCLING & 4.39 & Bicycle & ID 3 \\
TRANSIT\_RAIL & 8.29 & Rail & ID 87 (diesel)\\
TRANSIT\_METRO\_RAIL & 8.29 & Light rail transit & ID 94 (electric)\\
TRANSIT\_SUBWAY & 12.36 & Underground light rail & ID 100 (electric) \\
TRANSIT\_TRAM & 12.36 & Above ground light rail & ID 100 (electric) \\
TRANSIT\_MONORAIL & 8.29 & Monorail & ID 94 (electric) \\
TRANSIT\_HEAVY\_RAIL & 5.53 & Heavy rail & ID 93 (electric) \\
TRANSIT\_COMMUTER\_TRAIN & 21.57 & Commuter rail & ID 84 (diesel) \\
TRANSIT\_HIGH\_SPEED\_TRAIN & 6.35 & High speed train & ID 93 * $\approx$ 31/27 (based on source \cite[p. 35]{network-rail-comparison}) (electric) \\
TRANSIT\_BUS & 15.11 & Bus & ID 23 (from 2006)\\
TRANSIT\_INTERCITY\_BUS & 10.46 & Intercity bus & ID 31 (from 2006)\\
TRANSIT\_TROLLEYBUS & 10.58 & Trolleybus & ID 23 * $\approx$ 0.7 (based on source \cite[p. 101]{life-cycle-buses})\\
TRANSIT\_SHARE\_TAXI & 15.23 & Share taxi is a kind of bus with the ability to drop off and pick up passengers anywhere on its route & ID 66 (from 2006) \\
TRANSIT\_FERRY & 5 & Ferry & Default * 0.5, NO DATA (educated guess)\\
TRANSIT\_CARFERRY & 8 & Car ferry & Default * 0.8, NO DATA (educated guess)\\
TRANSIT\_CABLE\_CAR & 8 & A vehicle that operates on a cable, usually on the ground. Aerial cable cars may be of the type TRANSIT\_GONDOLA\_LIFT & Default * 0.5, NO DATA (educated guess)\\
TRANSIT\_GONDOLA\_LIFT & 8 & An aerial cable car & Default * 0.8, NO DATA (educated guess)\\
TRANSIT\_FUNICULAR & 4 & A vehicle that is pulled up a steep incline by a cable. A Funicular typically consists of two cars, with each car acting as a counterweight for the other & Default * 0.4, NO DATA (educated guess)\\
TRANSIT\_OTHER & 10 & All other vehicles will return this type & Default, NO DATA\\
\hline
\end{tabular}
\end{table}
\section{Indicator}
At first I had a normal RGB gradient between green and red, this gave a very nasty brownish color in the middle. (Figure \ref{fig:green-to-red-rgb})
This does not look very good the better option it to make the gradient by changing the hue from HSL/HSB colors. (Figure \ref{fig:green-to-red-hsb})
\begin{figure}[H]
	\centering	
	\includegraphics[width=\textwidth]{resources/green-to-red-rgb.png}
	\caption{The RGB gradient.}
	\label{fig:green-to-red-rgb}
\end{figure}
\begin{figure}[H]
	\centering	
	\includegraphics[width=\textwidth]{resources/green-to-red-hsb.png}
	\caption{The HSL/HSB gradient.}
	\label{fig:green-to-red-hsb}
\end{figure}

To move the yellow to the right or to the left we can use several functions to transform the hue value. (Figure \ref{fig:colors}) In the end I picked the almost squared root ($\text{x}^\text{0.45}$). This to make the difference between cycling (around 4.2-4.5) and walking (around 5) more apparent.

\begin{figure}[H]
	\centering
	\setlength\figureheight{8cm}
    	\setlength\figurewidth{0.9\linewidth}
	% This file was created by matlab2tikz.
% Minimal pgfplots version: 1.3
%
%The latest updates can be retrieved from
%  http://www.mathworks.com/matlabcentral/fileexchange/22022-matlab2tikz
%where you can also make suggestions and rate matlab2tikz.
%
\definecolor{mycolor1}{rgb}{0.00000,0.44700,0.74100}%
\definecolor{mycolor2}{rgb}{0.85000,0.32500,0.09800}%
\definecolor{mycolor3}{rgb}{0.92900,0.69400,0.12500}%
%
\begin{tikzpicture}

\begin{axis}[%
width=0.95092\figurewidth,
height=\figureheight,
at={(0\figurewidth,0\figureheight)},
scale only axis,
xmin=0,
xmax=35,
xlabel={Impact per KM},
ymin=0,
ymax=1,
ylabel={Color HSL value (0=120 degrees, 1=0 degrees)},
title style={font=\bfseries},
title={The indicator color transformation function options},
legend style={at={(0.97,0.03)},anchor=south east,legend cell align=left,align=left,draw=white!15!black}
]
\addplot [color=mycolor1,solid]
  table[row sep=crcr]{%
3	0\\
3.02802802802803	0\\
3.05605605605606	0\\
3.08408408408408	0\\
3.11211211211211	0\\
3.14014014014014	0\\
3.16816816816817	0\\
3.1961961961962	0\\
3.22422422422422	0\\
3.25225225225225	0\\
3.28028028028028	0\\
3.30830830830831	0\\
3.33633633633634	0\\
3.36436436436436	0\\
3.39239239239239	0\\
3.42042042042042	0\\
3.44844844844845	0\\
3.47647647647648	0\\
3.5045045045045	0\\
3.53253253253253	0\\
3.56056056056056	0\\
3.58858858858859	0\\
3.61661661661662	0\\
3.64464464464464	0\\
3.67267267267267	0\\
3.7007007007007	0\\
3.72872872872873	0\\
3.75675675675676	0\\
3.78478478478478	0\\
3.81281281281281	0\\
3.84084084084084	0\\
3.86886886886887	0\\
3.8968968968969	0\\
3.92492492492493	0\\
3.95295295295295	0\\
3.98098098098098	0\\
4.00900900900901	0.00034650034650036\\
4.03703703703704	0.00142450142450143\\
4.06506506506507	0.0025025025025025\\
4.09309309309309	0.00358050358050357\\
4.12112112112112	0.00465850465850464\\
4.14914914914915	0.00573650573650575\\
4.17717717717718	0.00681450681450682\\
4.20520520520521	0.00789250789250789\\
4.23323323323323	0.00897050897050896\\
4.26126126126126	0.01004851004851\\
4.28928928928929	0.0111265111265111\\
4.31731731731732	0.0122045122045122\\
4.34534534534535	0.0132825132825133\\
4.37337337337337	0.0143605143605143\\
4.4014014014014	0.0154385154385154\\
4.42942942942943	0.0165165165165165\\
4.45745745745746	0.0175945175945176\\
4.48548548548549	0.0186725186725187\\
4.51351351351351	0.0197505197505198\\
4.54154154154154	0.0208285208285208\\
4.56956956956957	0.0219065219065219\\
4.5975975975976	0.022984522984523\\
4.62562562562563	0.0240625240625241\\
4.65365365365365	0.0251405251405252\\
4.68168168168168	0.0262185262185262\\
4.70970970970971	0.0272965272965273\\
4.73773773773774	0.0283745283745284\\
4.76576576576577	0.0294525294525294\\
4.79379379379379	0.0305305305305305\\
4.82182182182182	0.0316085316085316\\
4.84984984984985	0.0326865326865327\\
4.87787787787788	0.0337645337645338\\
4.90590590590591	0.0348425348425348\\
4.93393393393393	0.0359205359205359\\
4.96196196196196	0.036998536998537\\
4.98998998998999	0.0380765380765381\\
5.01801801801802	0.0391545391545391\\
5.04604604604605	0.0402325402325402\\
5.07407407407407	0.0413105413105413\\
5.1021021021021	0.0423885423885424\\
5.13013013013013	0.0434665434665435\\
5.15815815815816	0.0445445445445446\\
5.18618618618619	0.0456225456225456\\
5.21421421421421	0.0467005467005467\\
5.24224224224224	0.0477785477785478\\
5.27027027027027	0.0488565488565489\\
5.2982982982983	0.04993454993455\\
5.32632632632633	0.051012551012551\\
5.35435435435435	0.0520905520905521\\
5.38238238238238	0.0531685531685532\\
5.41041041041041	0.0542465542465542\\
5.43843843843844	0.0553245553245553\\
5.46646646646647	0.0564025564025564\\
5.49449449449449	0.0574805574805575\\
5.52252252252252	0.0585585585585586\\
5.55055055055055	0.0596365596365596\\
5.57857857857858	0.0607145607145607\\
5.60660660660661	0.0617925617925618\\
5.63463463463463	0.0628705628705629\\
5.66266266266266	0.063948563948564\\
5.69069069069069	0.065026565026565\\
5.71871871871872	0.0661045661045661\\
5.74674674674675	0.0671825671825672\\
5.77477477477477	0.0682605682605683\\
5.8028028028028	0.0693385693385693\\
5.83083083083083	0.0704165704165704\\
5.85885885885886	0.0714945714945715\\
5.88688688688689	0.0725725725725726\\
5.91491491491491	0.0736505736505736\\
5.94294294294294	0.0747285747285747\\
5.97097097097097	0.0758065758065758\\
5.998998998999	0.0768845768845769\\
6.02702702702703	0.077962577962578\\
6.05505505505506	0.079040579040579\\
6.08308308308308	0.0801185801185801\\
6.11111111111111	0.0811965811965812\\
6.13913913913914	0.0822745822745823\\
6.16716716716717	0.0833525833525834\\
6.1951951951952	0.0844305844305844\\
6.22322322322322	0.0855085855085855\\
6.25125125125125	0.0865865865865866\\
6.27927927927928	0.0876645876645877\\
6.30730730730731	0.0887425887425888\\
6.33533533533534	0.0898205898205898\\
6.36336336336336	0.0908985908985909\\
6.39139139139139	0.091976591976592\\
6.41941941941942	0.0930545930545931\\
6.44744744744745	0.0941325941325941\\
6.47547547547548	0.0952105952105952\\
6.5035035035035	0.0962885962885963\\
6.53153153153153	0.0973665973665973\\
6.55955955955956	0.0984445984445985\\
6.58758758758759	0.0995225995225995\\
6.61561561561562	0.100600600600601\\
6.64364364364364	0.101678601678602\\
6.67167167167167	0.102756602756603\\
6.6996996996997	0.103834603834604\\
6.72772772772773	0.104912604912605\\
6.75575575575576	0.105990605990606\\
6.78378378378378	0.107068607068607\\
6.81181181181181	0.108146608146608\\
6.83983983983984	0.109224609224609\\
6.86786786786787	0.11030261030261\\
6.8958958958959	0.111380611380611\\
6.92392392392392	0.112458612458612\\
6.95195195195195	0.113536613536614\\
6.97997997997998	0.114614614614615\\
7.00800800800801	0.115692615692616\\
7.03603603603604	0.116770616770617\\
7.06406406406406	0.117848617848618\\
7.09209209209209	0.118926618926619\\
7.12012012012012	0.12000462000462\\
7.14814814814815	0.121082621082621\\
7.17617617617618	0.122160622160622\\
7.2042042042042	0.123238623238623\\
7.23223223223223	0.124316624316624\\
7.26026026026026	0.125394625394625\\
7.28828828828829	0.126472626472626\\
7.31631631631632	0.127550627550628\\
7.34434434434434	0.128628628628629\\
7.37237237237237	0.12970662970663\\
7.4004004004004	0.130784630784631\\
7.42842842842843	0.131862631862632\\
7.45645645645646	0.132940632940633\\
7.48448448448448	0.134018634018634\\
7.51251251251251	0.135096635096635\\
7.54054054054054	0.136174636174636\\
7.56856856856857	0.137252637252637\\
7.5965965965966	0.138330638330638\\
7.62462462462462	0.139408639408639\\
7.65265265265265	0.14048664048664\\
7.68068068068068	0.141564641564642\\
7.70870870870871	0.142642642642643\\
7.73673673673674	0.143720643720644\\
7.76476476476476	0.144798644798645\\
7.79279279279279	0.145876645876646\\
7.82082082082082	0.146954646954647\\
7.84884884884885	0.148032648032648\\
7.87687687687688	0.149110649110649\\
7.9049049049049	0.15018865018865\\
7.93293293293293	0.151266651266651\\
7.96096096096096	0.152344652344652\\
7.98898898898899	0.153422653422653\\
8.01701701701702	0.154500654500655\\
8.04504504504505	0.155578655578656\\
8.07307307307307	0.156656656656657\\
8.1011011011011	0.157734657734658\\
8.12912912912913	0.158812658812659\\
8.15715715715716	0.15989065989066\\
8.18518518518519	0.160968660968661\\
8.21321321321321	0.162046662046662\\
8.24124124124124	0.163124663124663\\
8.26926926926927	0.164202664202664\\
8.2972972972973	0.165280665280665\\
8.32532532532533	0.166358666358666\\
8.35335335335335	0.167436667436667\\
8.38138138138138	0.168514668514669\\
8.40940940940941	0.16959266959267\\
8.43743743743744	0.170670670670671\\
8.46546546546547	0.171748671748672\\
8.49349349349349	0.172826672826673\\
8.52152152152152	0.173904673904674\\
8.54954954954955	0.174982674982675\\
8.57757757757758	0.176060676060676\\
8.60560560560561	0.177138677138677\\
8.63363363363363	0.178216678216678\\
8.66166166166166	0.179294679294679\\
8.68968968968969	0.18037268037268\\
8.71771771771772	0.181450681450681\\
8.74574574574575	0.182528682528683\\
8.77377377377377	0.183606683606684\\
8.8018018018018	0.184684684684685\\
8.82982982982983	0.185762685762686\\
8.85785785785786	0.186840686840687\\
8.88588588588589	0.187918687918688\\
8.91391391391391	0.188996688996689\\
8.94194194194194	0.19007469007469\\
8.96996996996997	0.191152691152691\\
8.997997997998	0.192230692230692\\
9.02602602602603	0.193308693308693\\
9.05405405405405	0.194386694386694\\
9.08208208208208	0.195464695464695\\
9.11011011011011	0.196542696542697\\
9.13813813813814	0.197620697620698\\
9.16616616616617	0.198698698698699\\
9.19419419419419	0.1997766997767\\
9.22222222222222	0.200854700854701\\
9.25025025025025	0.201932701932702\\
9.27827827827828	0.203010703010703\\
9.30630630630631	0.204088704088704\\
9.33433433433433	0.205166705166705\\
9.36236236236236	0.206244706244706\\
9.39039039039039	0.207322707322707\\
9.41841841841842	0.208400708400708\\
9.44644644644645	0.20947870947871\\
9.47447447447447	0.210556710556711\\
9.5025025025025	0.211634711634712\\
9.53053053053053	0.212712712712713\\
9.55855855855856	0.213790713790714\\
9.58658658658659	0.214868714868715\\
9.61461461461461	0.215946715946716\\
9.64264264264264	0.217024717024717\\
9.67067067067067	0.218102718102718\\
9.6986986986987	0.219180719180719\\
9.72672672672673	0.22025872025872\\
9.75475475475475	0.221336721336721\\
9.78278278278278	0.222414722414722\\
9.81081081081081	0.223492723492723\\
9.83883883883884	0.224570724570725\\
9.86686686686687	0.225648725648726\\
9.89489489489489	0.226726726726727\\
9.92292292292292	0.227804727804728\\
9.95095095095095	0.228882728882729\\
9.97897897897898	0.22996072996073\\
10.007007007007	0.231038731038731\\
10.035035035035	0.232116732116732\\
10.0630630630631	0.233194733194733\\
10.0910910910911	0.234272734272734\\
10.1191191191191	0.235350735350735\\
10.1471471471471	0.236428736428736\\
10.1751751751752	0.237506737506738\\
10.2032032032032	0.238584738584739\\
10.2312312312312	0.23966273966274\\
10.2592592592593	0.240740740740741\\
10.2872872872873	0.241818741818742\\
10.3153153153153	0.242896742896743\\
10.3433433433433	0.243974743974744\\
10.3713713713714	0.245052745052745\\
10.3993993993994	0.246130746130746\\
10.4274274274274	0.247208747208747\\
10.4554554554555	0.248286748286748\\
10.4834834834835	0.249364749364749\\
10.5115115115115	0.25044275044275\\
10.5395395395395	0.251520751520752\\
10.5675675675676	0.252598752598753\\
10.5955955955956	0.253676753676754\\
10.6236236236236	0.254754754754755\\
10.6516516516517	0.255832755832756\\
10.6796796796797	0.256910756910757\\
10.7077077077077	0.257988757988758\\
10.7357357357357	0.259066759066759\\
10.7637637637638	0.26014476014476\\
10.7917917917918	0.261222761222761\\
10.8198198198198	0.262300762300762\\
10.8478478478478	0.263378763378763\\
10.8758758758759	0.264456764456764\\
10.9039039039039	0.265534765534766\\
10.9319319319319	0.266612766612767\\
10.95995995996	0.267690767690768\\
10.987987987988	0.268768768768769\\
11.016016016016	0.26984676984677\\
11.044044044044	0.270924770924771\\
11.0720720720721	0.272002772002772\\
11.1001001001001	0.273080773080773\\
11.1281281281281	0.274158774158774\\
11.1561561561562	0.275236775236775\\
11.1841841841842	0.276314776314776\\
11.2122122122122	0.277392777392777\\
11.2402402402402	0.278470778470778\\
11.2682682682683	0.27954877954878\\
11.2962962962963	0.280626780626781\\
11.3243243243243	0.281704781704782\\
11.3523523523524	0.282782782782783\\
11.3803803803804	0.283860783860784\\
11.4084084084084	0.284938784938785\\
11.4364364364364	0.286016786016786\\
11.4644644644645	0.287094787094787\\
11.4924924924925	0.288172788172788\\
11.5205205205205	0.289250789250789\\
11.5485485485485	0.29032879032879\\
11.5765765765766	0.291406791406791\\
11.6046046046046	0.292484792484792\\
11.6326326326326	0.293562793562794\\
11.6606606606607	0.294640794640795\\
11.6886886886887	0.295718795718796\\
11.7167167167167	0.296796796796797\\
11.7447447447447	0.297874797874798\\
11.7727727727728	0.298952798952799\\
11.8008008008008	0.3000308000308\\
11.8288288288288	0.301108801108801\\
11.8568568568569	0.302186802186802\\
11.8848848848849	0.303264803264803\\
11.9129129129129	0.304342804342804\\
11.9409409409409	0.305420805420805\\
11.968968968969	0.306498806498807\\
11.996996996997	0.307576807576808\\
12.025025025025	0.308654808654809\\
12.0530530530531	0.30973280973281\\
12.0810810810811	0.310810810810811\\
12.1091091091091	0.311888811888812\\
12.1371371371371	0.312966812966813\\
12.1651651651652	0.314044814044814\\
12.1931931931932	0.315122815122815\\
12.2212212212212	0.316200816200816\\
12.2492492492493	0.317278817278817\\
12.2772772772773	0.318356818356818\\
12.3053053053053	0.319434819434819\\
12.3333333333333	0.320512820512821\\
12.3613613613614	0.321590821590822\\
12.3893893893894	0.322668822668823\\
12.4174174174174	0.323746823746824\\
12.4454454454454	0.324824824824825\\
12.4734734734735	0.325902825902826\\
12.5015015015015	0.326980826980827\\
12.5295295295295	0.328058828058828\\
12.5575575575576	0.329136829136829\\
12.5855855855856	0.33021483021483\\
12.6136136136136	0.331292831292831\\
12.6416416416416	0.332370832370832\\
12.6696696696697	0.333448833448833\\
12.6976976976977	0.334526834526835\\
12.7257257257257	0.335604835604836\\
12.7537537537538	0.336682836682837\\
12.7817817817818	0.337760837760838\\
12.8098098098098	0.338838838838839\\
12.8378378378378	0.33991683991684\\
12.8658658658659	0.340994840994841\\
12.8938938938939	0.342072842072842\\
12.9219219219219	0.343150843150843\\
12.94994994995	0.344228844228844\\
12.977977977978	0.345306845306845\\
13.006006006006	0.346384846384846\\
13.034034034034	0.347462847462848\\
13.0620620620621	0.348540848540848\\
13.0900900900901	0.34961884961885\\
13.1181181181181	0.350696850696851\\
13.1461461461461	0.351774851774852\\
13.1741741741742	0.352852852852853\\
13.2022022022022	0.353930853930854\\
13.2302302302302	0.355008855008855\\
13.2582582582583	0.356086856086856\\
13.2862862862863	0.357164857164857\\
13.3143143143143	0.358242858242858\\
13.3423423423423	0.359320859320859\\
13.3703703703704	0.36039886039886\\
13.3983983983984	0.361476861476861\\
13.4264264264264	0.362554862554863\\
13.4544544544545	0.363632863632864\\
13.4824824824825	0.364710864710865\\
13.5105105105105	0.365788865788866\\
13.5385385385385	0.366866866866867\\
13.5665665665666	0.367944867944868\\
13.5945945945946	0.369022869022869\\
13.6226226226226	0.37010087010087\\
13.6506506506507	0.371178871178871\\
13.6786786786787	0.372256872256872\\
13.7067067067067	0.373334873334873\\
13.7347347347347	0.374412874412874\\
13.7627627627628	0.375490875490876\\
13.7907907907908	0.376568876568877\\
13.8188188188188	0.377646877646878\\
13.8468468468468	0.378724878724879\\
13.8748748748749	0.37980287980288\\
13.9029029029029	0.380880880880881\\
13.9309309309309	0.381958881958882\\
13.958958958959	0.383036883036883\\
13.986986986987	0.384114884114884\\
14.015015015015	0.385192885192885\\
14.043043043043	0.386270886270886\\
14.0710710710711	0.387348887348887\\
14.0990990990991	0.388426888426888\\
14.1271271271271	0.389504889504889\\
14.1551551551552	0.390582890582891\\
14.1831831831832	0.391660891660892\\
14.2112112112112	0.392738892738893\\
14.2392392392392	0.393816893816894\\
14.2672672672673	0.394894894894895\\
14.2952952952953	0.395972895972896\\
14.3233233233233	0.397050897050897\\
14.3513513513514	0.398128898128898\\
14.3793793793794	0.399206899206899\\
14.4074074074074	0.4002849002849\\
14.4354354354354	0.401362901362901\\
14.4634634634635	0.402440902440902\\
14.4914914914915	0.403518903518904\\
14.5195195195195	0.404596904596905\\
14.5475475475475	0.405674905674906\\
14.5755755755756	0.406752906752907\\
14.6036036036036	0.407830907830908\\
14.6316316316316	0.408908908908909\\
14.6596596596597	0.40998690998691\\
14.6876876876877	0.411064911064911\\
14.7157157157157	0.412142912142912\\
14.7437437437437	0.413220913220913\\
14.7717717717718	0.414298914298914\\
14.7997997997998	0.415376915376915\\
14.8278278278278	0.416454916454917\\
14.8558558558559	0.417532917532918\\
14.8838838838839	0.418610918610919\\
14.9119119119119	0.41968891968892\\
14.9399399399399	0.420766920766921\\
14.967967967968	0.421844921844922\\
14.995995995996	0.422922922922923\\
15.024024024024	0.424000924000924\\
15.0520520520521	0.425078925078925\\
15.0800800800801	0.426156926156926\\
15.1081081081081	0.427234927234927\\
15.1361361361361	0.428312928312928\\
15.1641641641642	0.429390929390929\\
15.1921921921922	0.43046893046893\\
15.2202202202202	0.431546931546932\\
15.2482482482482	0.432624932624933\\
15.2762762762763	0.433702933702934\\
15.3043043043043	0.434780934780935\\
15.3323323323323	0.435858935858936\\
15.3603603603604	0.436936936936937\\
15.3883883883884	0.438014938014938\\
15.4164164164164	0.439092939092939\\
15.4444444444444	0.44017094017094\\
15.4724724724725	0.441248941248941\\
15.5005005005005	0.442326942326942\\
15.5285285285285	0.443404943404943\\
15.5565565565566	0.444482944482944\\
15.5845845845846	0.445560945560946\\
15.6126126126126	0.446638946638947\\
15.6406406406406	0.447716947716948\\
15.6686686686687	0.448794948794949\\
15.6966966966967	0.44987294987295\\
15.7247247247247	0.450950950950951\\
15.7527527527528	0.452028952028952\\
15.7807807807808	0.453106953106953\\
15.8088088088088	0.454184954184954\\
15.8368368368368	0.455262955262955\\
15.8648648648649	0.456340956340956\\
15.8928928928929	0.457418957418957\\
15.9209209209209	0.458496958496958\\
15.9489489489489	0.45957495957496\\
15.976976976977	0.460652960652961\\
16.005005005005	0.461730961730962\\
16.033033033033	0.462808962808963\\
16.0610610610611	0.463886963886964\\
16.0890890890891	0.464964964964965\\
16.1171171171171	0.466042966042966\\
16.1451451451451	0.467120967120967\\
16.1731731731732	0.468198968198968\\
16.2012012012012	0.469276969276969\\
16.2292292292292	0.47035497035497\\
16.2572572572573	0.471432971432971\\
16.2852852852853	0.472510972510972\\
16.3133133133133	0.473588973588974\\
16.3413413413413	0.474666974666975\\
16.3693693693694	0.475744975744976\\
16.3973973973974	0.476822976822977\\
16.4254254254254	0.477900977900978\\
16.4534534534535	0.478978978978979\\
16.4814814814815	0.48005698005698\\
16.5095095095095	0.481134981134981\\
16.5375375375375	0.482212982212982\\
16.5655655655656	0.483290983290983\\
16.5935935935936	0.484368984368984\\
16.6216216216216	0.485446985446985\\
16.6496496496497	0.486524986524987\\
16.6776776776777	0.487602987602988\\
16.7057057057057	0.488680988680989\\
16.7337337337337	0.48975898975899\\
16.7617617617618	0.490836990836991\\
16.7897897897898	0.491914991914992\\
16.8178178178178	0.492992992992993\\
16.8458458458458	0.494070994070994\\
16.8738738738739	0.495148995148995\\
16.9019019019019	0.496226996226996\\
16.9299299299299	0.497304997304997\\
16.957957957958	0.498382998382998\\
16.985985985986	0.499460999460999\\
17.014014014014	0.500539000539001\\
17.042042042042	0.501617001617002\\
17.0700700700701	0.502695002695003\\
17.0980980980981	0.503773003773004\\
17.1261261261261	0.504851004851005\\
17.1541541541542	0.505929005929006\\
17.1821821821822	0.507007007007007\\
17.2102102102102	0.508085008085008\\
17.2382382382382	0.509163009163009\\
17.2662662662663	0.51024101024101\\
17.2942942942943	0.511319011319011\\
17.3223223223223	0.512397012397012\\
17.3503503503504	0.513475013475013\\
17.3783783783784	0.514553014553015\\
17.4064064064064	0.515631015631016\\
17.4344344344344	0.516709016709017\\
17.4624624624625	0.517787017787018\\
17.4904904904905	0.518865018865019\\
17.5185185185185	0.51994301994302\\
17.5465465465465	0.521021021021021\\
17.5745745745746	0.522099022099022\\
17.6026026026026	0.523177023177023\\
17.6306306306306	0.524255024255024\\
17.6586586586587	0.525333025333025\\
17.6866866866867	0.526411026411026\\
17.7147147147147	0.527489027489028\\
17.7427427427427	0.528567028567029\\
17.7707707707708	0.52964502964503\\
17.7987987987988	0.530723030723031\\
17.8268268268268	0.531801031801032\\
17.8548548548549	0.532879032879033\\
17.8828828828829	0.533957033957034\\
17.9109109109109	0.535035035035035\\
17.9389389389389	0.536113036113036\\
17.966966966967	0.537191037191037\\
17.994994994995	0.538269038269038\\
18.023023023023	0.539347039347039\\
18.0510510510511	0.54042504042504\\
18.0790790790791	0.541503041503042\\
18.1071071071071	0.542581042581042\\
18.1351351351351	0.543659043659044\\
18.1631631631632	0.544737044737045\\
18.1911911911912	0.545815045815046\\
18.2192192192192	0.546893046893047\\
18.2472472472472	0.547971047971048\\
18.2752752752753	0.549049049049049\\
18.3033033033033	0.55012705012705\\
18.3313313313313	0.551205051205051\\
18.3593593593594	0.552283052283052\\
18.3873873873874	0.553361053361053\\
18.4154154154154	0.554439054439055\\
18.4434434434434	0.555517055517055\\
18.4714714714715	0.556595056595057\\
18.4994994994995	0.557673057673058\\
18.5275275275275	0.558751058751059\\
18.5555555555556	0.55982905982906\\
18.5835835835836	0.560907060907061\\
18.6116116116116	0.561985061985062\\
18.6396396396396	0.563063063063063\\
18.6676676676677	0.564141064141064\\
18.6956956956957	0.565219065219065\\
18.7237237237237	0.566297066297066\\
18.7517517517518	0.567375067375067\\
18.7797797797798	0.568453068453068\\
18.8078078078078	0.56953106953107\\
18.8358358358358	0.570609070609071\\
18.8638638638639	0.571687071687072\\
18.8918918918919	0.572765072765073\\
18.9199199199199	0.573843073843074\\
18.9479479479479	0.574921074921075\\
18.975975975976	0.575999075999076\\
19.004004004004	0.577077077077077\\
19.032032032032	0.578155078155078\\
19.0600600600601	0.579233079233079\\
19.0880880880881	0.58031108031108\\
19.1161161161161	0.581389081389081\\
19.1441441441441	0.582467082467082\\
19.1721721721722	0.583545083545084\\
19.2002002002002	0.584623084623085\\
19.2282282282282	0.585701085701086\\
19.2562562562563	0.586779086779087\\
19.2842842842843	0.587857087857088\\
19.3123123123123	0.588935088935089\\
19.3403403403403	0.59001309001309\\
19.3683683683684	0.591091091091091\\
19.3963963963964	0.592169092169092\\
19.4244244244244	0.593247093247093\\
19.4524524524525	0.594325094325094\\
19.4804804804805	0.595403095403095\\
19.5085085085085	0.596481096481096\\
19.5365365365365	0.597559097559098\\
19.5645645645646	0.598637098637099\\
19.5925925925926	0.5997150997151\\
19.6206206206206	0.600793100793101\\
19.6486486486486	0.601871101871102\\
19.6766766766767	0.602949102949103\\
19.7047047047047	0.604027104027104\\
19.7327327327327	0.605105105105105\\
19.7607607607608	0.606183106183106\\
19.7887887887888	0.607261107261107\\
19.8168168168168	0.608339108339108\\
19.8448448448448	0.609417109417109\\
19.8728728728729	0.61049511049511\\
19.9009009009009	0.611573111573112\\
19.9289289289289	0.612651112651113\\
19.956956956957	0.613729113729114\\
19.984984984985	0.614807114807115\\
20.013013013013	0.615885115885116\\
20.041041041041	0.616963116963117\\
20.0690690690691	0.618041118041118\\
20.0970970970971	0.619119119119119\\
20.1251251251251	0.62019712019712\\
20.1531531531532	0.621275121275121\\
20.1811811811812	0.622353122353122\\
20.2092092092092	0.623431123431123\\
20.2372372372372	0.624509124509125\\
20.2652652652653	0.625587125587126\\
20.2932932932933	0.626665126665127\\
20.3213213213213	0.627743127743128\\
20.3493493493494	0.628821128821129\\
20.3773773773774	0.62989912989913\\
20.4054054054054	0.630977130977131\\
20.4334334334334	0.632055132055132\\
20.4614614614615	0.633133133133133\\
20.4894894894895	0.634211134211134\\
20.5175175175175	0.635289135289135\\
20.5455455455455	0.636367136367136\\
20.5735735735736	0.637445137445137\\
20.6016016016016	0.638523138523138\\
20.6296296296296	0.63960113960114\\
20.6576576576577	0.640679140679141\\
20.6856856856857	0.641757141757142\\
20.7137137137137	0.642835142835143\\
20.7417417417417	0.643913143913144\\
20.7697697697698	0.644991144991145\\
20.7977977977978	0.646069146069146\\
20.8258258258258	0.647147147147147\\
20.8538538538539	0.648225148225148\\
20.8818818818819	0.649303149303149\\
20.9099099099099	0.65038115038115\\
20.9379379379379	0.651459151459151\\
20.965965965966	0.652537152537153\\
20.993993993994	0.653615153615154\\
21.022022022022	0.654693154693155\\
21.0500500500501	0.655771155771156\\
21.0780780780781	0.656849156849157\\
21.1061061061061	0.657927157927158\\
21.1341341341341	0.659005159005159\\
21.1621621621622	0.66008316008316\\
21.1901901901902	0.661161161161161\\
21.2182182182182	0.662239162239162\\
21.2462462462462	0.663317163317163\\
21.2742742742743	0.664395164395164\\
21.3023023023023	0.665473165473165\\
21.3303303303303	0.666551166551167\\
21.3583583583584	0.667629167629168\\
21.3863863863864	0.668707168707169\\
21.4144144144144	0.66978516978517\\
21.4424424424424	0.670863170863171\\
21.4704704704705	0.671941171941172\\
21.4984984984985	0.673019173019173\\
21.5265265265265	0.674097174097174\\
21.5545545545546	0.675175175175175\\
21.5825825825826	0.676253176253176\\
21.6106106106106	0.677331177331177\\
21.6386386386386	0.678409178409178\\
21.6666666666667	0.67948717948718\\
21.6946946946947	0.68056518056518\\
21.7227227227227	0.681643181643182\\
21.7507507507508	0.682721182721183\\
21.7787787787788	0.683799183799184\\
21.8068068068068	0.684877184877185\\
21.8348348348348	0.685955185955186\\
21.8628628628629	0.687033187033187\\
21.8908908908909	0.688111188111188\\
21.9189189189189	0.689189189189189\\
21.9469469469469	0.69026719026719\\
21.974974974975	0.691345191345191\\
22.003003003003	0.692423192423192\\
22.031031031031	0.693501193501193\\
22.0590590590591	0.694579194579195\\
22.0870870870871	0.695657195657196\\
22.1151151151151	0.696735196735197\\
22.1431431431431	0.697813197813198\\
22.1711711711712	0.698891198891199\\
22.1991991991992	0.6999691999692\\
22.2272272272272	0.701047201047201\\
22.2552552552553	0.702125202125202\\
22.2832832832833	0.703203203203203\\
22.3113113113113	0.704281204281204\\
22.3393393393393	0.705359205359205\\
22.3673673673674	0.706437206437206\\
22.3953953953954	0.707515207515208\\
22.4234234234234	0.708593208593209\\
22.4514514514515	0.70967120967121\\
22.4794794794795	0.710749210749211\\
22.5075075075075	0.711827211827212\\
22.5355355355355	0.712905212905213\\
22.5635635635636	0.713983213983214\\
22.5915915915916	0.715061215061215\\
22.6196196196196	0.716139216139216\\
22.6476476476476	0.717217217217217\\
22.6756756756757	0.718295218295218\\
22.7037037037037	0.719373219373219\\
22.7317317317317	0.72045122045122\\
22.7597597597598	0.721529221529222\\
22.7877877877878	0.722607222607223\\
22.8158158158158	0.723685223685224\\
22.8438438438438	0.724763224763225\\
22.8718718718719	0.725841225841226\\
22.8998998998999	0.726919226919227\\
22.9279279279279	0.727997227997228\\
22.955955955956	0.729075229075229\\
22.983983983984	0.73015323015323\\
23.012012012012	0.731231231231231\\
23.04004004004	0.732309232309232\\
23.0680680680681	0.733387233387233\\
23.0960960960961	0.734465234465234\\
23.1241241241241	0.735543235543236\\
23.1521521521522	0.736621236621237\\
23.1801801801802	0.737699237699238\\
23.2082082082082	0.738777238777239\\
23.2362362362362	0.73985523985524\\
23.2642642642643	0.740933240933241\\
23.2922922922923	0.742011242011242\\
23.3203203203203	0.743089243089243\\
23.3483483483483	0.744167244167244\\
23.3763763763764	0.745245245245245\\
23.4044044044044	0.746323246323246\\
23.4324324324324	0.747401247401247\\
23.4604604604605	0.748479248479249\\
23.4884884884885	0.74955724955725\\
23.5165165165165	0.750635250635251\\
23.5445445445445	0.751713251713252\\
23.5725725725726	0.752791252791253\\
23.6006006006006	0.753869253869254\\
23.6286286286286	0.754947254947255\\
23.6566566566567	0.756025256025256\\
23.6846846846847	0.757103257103257\\
23.7127127127127	0.758181258181258\\
23.7407407407407	0.759259259259259\\
23.7687687687688	0.76033726033726\\
23.7967967967968	0.761415261415262\\
23.8248248248248	0.762493262493262\\
23.8528528528529	0.763571263571263\\
23.8808808808809	0.764649264649265\\
23.9089089089089	0.765727265727266\\
23.9369369369369	0.766805266805267\\
23.964964964965	0.767883267883268\\
23.992992992993	0.768961268961269\\
24.021021021021	0.77003927003927\\
24.049049049049	0.771117271117271\\
24.0770770770771	0.772195272195272\\
24.1051051051051	0.773273273273273\\
24.1331331331331	0.774351274351274\\
24.1611611611612	0.775429275429275\\
24.1891891891892	0.776507276507276\\
24.2172172172172	0.777585277585278\\
24.2452452452452	0.778663278663279\\
24.2732732732733	0.77974127974128\\
24.3013013013013	0.780819280819281\\
24.3293293293293	0.781897281897282\\
24.3573573573574	0.782975282975283\\
24.3853853853854	0.784053284053284\\
24.4134134134134	0.785131285131285\\
24.4414414414414	0.786209286209286\\
24.4694694694695	0.787287287287287\\
24.4974974974975	0.788365288365288\\
24.5255255255255	0.789443289443289\\
24.5535535535536	0.790521290521291\\
24.5815815815816	0.791599291599292\\
24.6096096096096	0.792677292677293\\
24.6376376376376	0.793755293755294\\
24.6656656656657	0.794833294833295\\
24.6936936936937	0.795911295911296\\
24.7217217217217	0.796989296989297\\
24.7497497497498	0.798067298067298\\
24.7777777777778	0.799145299145299\\
24.8058058058058	0.8002233002233\\
24.8338338338338	0.801301301301301\\
24.8618618618619	0.802379302379302\\
24.8898898898899	0.803457303457303\\
24.9179179179179	0.804535304535305\\
24.9459459459459	0.805613305613306\\
24.973973973974	0.806691306691307\\
25.002002002002	0.807769307769308\\
25.03003003003	0.808847308847309\\
25.0580580580581	0.80992530992531\\
25.0860860860861	0.811003311003311\\
25.1141141141141	0.812081312081312\\
25.1421421421421	0.813159313159313\\
25.1701701701702	0.814237314237314\\
25.1981981981982	0.815315315315315\\
25.2262262262262	0.816393316393316\\
25.2542542542543	0.817471317471317\\
25.2822822822823	0.818549318549319\\
25.3103103103103	0.81962731962732\\
25.3383383383383	0.820705320705321\\
25.3663663663664	0.821783321783322\\
25.3943943943944	0.822861322861323\\
25.4224224224224	0.823939323939324\\
25.4504504504505	0.825017325017325\\
25.4784784784785	0.826095326095326\\
25.5065065065065	0.827173327173327\\
25.5345345345345	0.828251328251328\\
25.5625625625626	0.829329329329329\\
25.5905905905906	0.83040733040733\\
25.6186186186186	0.831485331485332\\
25.6466466466466	0.832563332563333\\
25.6746746746747	0.833641333641334\\
25.7027027027027	0.834719334719335\\
25.7307307307307	0.835797335797336\\
25.7587587587588	0.836875336875337\\
25.7867867867868	0.837953337953338\\
25.8148148148148	0.839031339031339\\
25.8428428428428	0.84010934010934\\
25.8708708708709	0.841187341187341\\
25.8988988988989	0.842265342265342\\
25.9269269269269	0.843343343343343\\
25.954954954955	0.844421344421344\\
25.982982982983	0.845499345499345\\
26.011011011011	0.846577346577347\\
26.039039039039	0.847655347655348\\
26.0670670670671	0.848733348733349\\
26.0950950950951	0.84981134981135\\
26.1231231231231	0.850889350889351\\
26.1511511511512	0.851967351967352\\
26.1791791791792	0.853045353045353\\
26.2072072072072	0.854123354123354\\
26.2352352352352	0.855201355201355\\
26.2632632632633	0.856279356279356\\
26.2912912912913	0.857357357357357\\
26.3193193193193	0.858435358435358\\
26.3473473473473	0.85951335951336\\
26.3753753753754	0.860591360591361\\
26.4034034034034	0.861669361669362\\
26.4314314314314	0.862747362747363\\
26.4594594594595	0.863825363825364\\
26.4874874874875	0.864903364903365\\
26.5155155155155	0.865981365981366\\
26.5435435435435	0.867059367059367\\
26.5715715715716	0.868137368137368\\
26.5995995995996	0.869215369215369\\
26.6276276276276	0.87029337029337\\
26.6556556556557	0.871371371371371\\
26.6836836836837	0.872449372449372\\
26.7117117117117	0.873527373527374\\
26.7397397397397	0.874605374605375\\
26.7677677677678	0.875683375683376\\
26.7957957957958	0.876761376761377\\
26.8238238238238	0.877839377839378\\
26.8518518518519	0.878917378917379\\
26.8798798798799	0.87999537999538\\
26.9079079079079	0.881073381073381\\
26.9359359359359	0.882151382151382\\
26.963963963964	0.883229383229383\\
26.991991991992	0.884307384307384\\
27.02002002002	0.885385385385385\\
27.048048048048	0.886463386463387\\
27.0760760760761	0.887541387541388\\
27.1041041041041	0.888619388619389\\
27.1321321321321	0.88969738969739\\
27.1601601601602	0.890775390775391\\
27.1881881881882	0.891853391853392\\
27.2162162162162	0.892931392931393\\
27.2442442442442	0.894009394009394\\
27.2722722722723	0.895087395087395\\
27.3003003003003	0.896165396165396\\
27.3283283283283	0.897243397243397\\
27.3563563563564	0.898321398321398\\
27.3843843843844	0.899399399399399\\
27.4124124124124	0.9004774004774\\
27.4404404404404	0.901555401555402\\
27.4684684684685	0.902633402633403\\
27.4964964964965	0.903711403711404\\
27.5245245245245	0.904789404789405\\
27.5525525525526	0.905867405867406\\
27.5805805805806	0.906945406945407\\
27.6086086086086	0.908023408023408\\
27.6366366366366	0.909101409101409\\
27.6646646646647	0.91017941017941\\
27.6926926926927	0.911257411257411\\
27.7207207207207	0.912335412335412\\
27.7487487487487	0.913413413413413\\
27.7767767767768	0.914491414491415\\
27.8048048048048	0.915569415569416\\
27.8328328328328	0.916647416647417\\
27.8608608608609	0.917725417725418\\
27.8888888888889	0.918803418803419\\
27.9169169169169	0.91988141988142\\
27.9449449449449	0.920959420959421\\
27.972972972973	0.922037422037422\\
28.001001001001	0.923115423115423\\
28.029029029029	0.924193424193424\\
28.0570570570571	0.925271425271425\\
28.0850850850851	0.926349426349426\\
28.1131131131131	0.927427427427427\\
28.1411411411411	0.928505428505429\\
28.1691691691692	0.92958342958343\\
28.1971971971972	0.930661430661431\\
28.2252252252252	0.931739431739432\\
28.2532532532533	0.932817432817433\\
28.2812812812813	0.933895433895434\\
28.3093093093093	0.934973434973435\\
28.3373373373373	0.936051436051436\\
28.3653653653654	0.937129437129437\\
28.3933933933934	0.938207438207438\\
28.4214214214214	0.939285439285439\\
28.4494494494495	0.94036344036344\\
28.4774774774775	0.941441441441441\\
28.5055055055055	0.942519442519443\\
28.5335335335335	0.943597443597444\\
28.5615615615616	0.944675444675445\\
28.5895895895896	0.945753445753446\\
28.6176176176176	0.946831446831447\\
28.6456456456456	0.947909447909448\\
28.6736736736737	0.948987448987449\\
28.7017017017017	0.95006545006545\\
28.7297297297297	0.951143451143451\\
28.7577577577578	0.952221452221452\\
28.7857857857858	0.953299453299453\\
28.8138138138138	0.954377454377454\\
28.8418418418418	0.955455455455455\\
28.8698698698699	0.956533456533457\\
28.8978978978979	0.957611457611458\\
28.9259259259259	0.958689458689459\\
28.953953953954	0.95976745976746\\
28.981981981982	0.960845460845461\\
29.01001001001	0.961923461923462\\
29.038038038038	0.963001463001463\\
29.0660660660661	0.964079464079464\\
29.0940940940941	0.965157465157465\\
29.1221221221221	0.966235466235466\\
29.1501501501502	0.967313467313467\\
29.1781781781782	0.968391468391468\\
29.2062062062062	0.96946946946947\\
29.2342342342342	0.97054747054747\\
29.2622622622623	0.971625471625472\\
29.2902902902903	0.972703472703473\\
29.3183183183183	0.973781473781474\\
29.3463463463463	0.974859474859475\\
29.3743743743744	0.975937475937476\\
29.4024024024024	0.977015477015477\\
29.4304304304304	0.978093478093478\\
29.4584584584585	0.979171479171479\\
29.4864864864865	0.98024948024948\\
29.5145145145145	0.981327481327481\\
29.5425425425425	0.982405482405482\\
29.5705705705706	0.983483483483483\\
29.5985985985986	0.984561484561485\\
29.6266266266266	0.985639485639486\\
29.6546546546547	0.986717486717487\\
29.6826826826827	0.987795487795488\\
29.7107107107107	0.988873488873489\\
29.7387387387387	0.98995148995149\\
29.7667667667668	0.991029491029491\\
29.7947947947948	0.992107492107492\\
29.8228228228228	0.993185493185493\\
29.8508508508509	0.994263494263494\\
29.8788788788789	0.995341495341495\\
29.9069069069069	0.996419496419496\\
29.9349349349349	0.997497497497497\\
29.962962962963	0.998575498575499\\
29.990990990991	0.9996534996535\\
30.019019019019	1\\
30.047047047047	1\\
30.0750750750751	1\\
30.1031031031031	1\\
30.1311311311311	1\\
30.1591591591592	1\\
30.1871871871872	1\\
30.2152152152152	1\\
30.2432432432432	1\\
30.2712712712713	1\\
30.2992992992993	1\\
30.3273273273273	1\\
30.3553553553554	1\\
30.3833833833834	1\\
30.4114114114114	1\\
30.4394394394394	1\\
30.4674674674675	1\\
30.4954954954955	1\\
30.5235235235235	1\\
30.5515515515516	1\\
30.5795795795796	1\\
30.6076076076076	1\\
30.6356356356356	1\\
30.6636636636637	1\\
30.6916916916917	1\\
30.7197197197197	1\\
30.7477477477477	1\\
30.7757757757758	1\\
30.8038038038038	1\\
30.8318318318318	1\\
30.8598598598599	1\\
30.8878878878879	1\\
30.9159159159159	1\\
30.9439439439439	1\\
30.971971971972	1\\
31	1\\
};
\addlegendentry{Linear};

\addplot [color=mycolor2,solid]
  table[row sep=crcr]{%
3	0\\
3.02802802802803	0\\
3.05605605605606	0\\
3.08408408408408	0\\
3.11211211211211	0\\
3.14014014014014	0\\
3.16816816816817	0\\
3.1961961961962	0\\
3.22422422422422	0\\
3.25225225225225	0\\
3.28028028028028	0\\
3.30830830830831	0\\
3.33633633633634	0\\
3.36436436436436	0\\
3.39239239239239	0\\
3.42042042042042	0\\
3.44844844844845	0\\
3.47647647647648	0\\
3.5045045045045	0\\
3.53253253253253	0\\
3.56056056056056	0\\
3.58858858858859	0\\
3.61661661661662	0\\
3.64464464464464	0\\
3.67267267267267	0\\
3.7007007007007	0\\
3.72872872872873	0\\
3.75675675675676	0\\
3.78478478478478	0\\
3.81281281281281	0\\
3.84084084084084	0\\
3.86886886886887	0\\
3.8968968968969	0\\
3.92492492492493	0\\
3.95295295295295	0\\
3.98098098098098	0\\
4.00900900900901	0.0186145197762489\\
4.03703703703704	0.0377425678048199\\
4.06506506506507	0.0500250187656387\\
4.09309309309309	0.059837309268579\\
4.12112112112112	0.0682532391795777\\
4.14914914914915	0.0757397236363175\\
4.17717717717718	0.082550026132684\\
4.20520520520521	0.0888397877783817\\
4.23323323323323	0.0947127708944732\\
4.26126126126126	0.100242256800763\\
4.28928928928929	0.105482278732075\\
4.31731731731732	0.110474034073678\\
4.34534534534535	0.115249786474914\\
4.37337337337337	0.119835363563993\\
4.4014014014014	0.124251822676834\\
4.42942942942943	0.128516600159343\\
4.45745745745746	0.132644327411758\\
4.48548548548549	0.136647424683082\\
4.51351351351351	0.140536542402749\\
4.54154154154154	0.14432089532885\\
4.56956956956957	0.148008519709245\\
4.5975975975976	0.151606474085123\\
4.62562562562563	0.155120998135404\\
4.65365365365365	0.158557639804978\\
4.68168168168168	0.161921358129575\\
4.70970970970971	0.165216607205593\\
4.73773773773774	0.168447405365973\\
4.76576576576577	0.171617392628281\\
4.79379379379379	0.174729878757271\\
4.82182182182182	0.177787883750642\\
4.84984984984985	0.180794172158653\\
4.87787787787788	0.183751282347998\\
4.90590590590591	0.186661551591469\\
4.93393393393393	0.18952713768887\\
4.96196196196196	0.192350037687901\\
4.98998998998999	0.195132104166736\\
5.01801801801802	0.197875059455554\\
5.04604604604605	0.200580508107194\\
5.07407407407407	0.203249947873404\\
5.1021021021021	0.205884779399893\\
5.13013013013013	0.208486314818368\\
5.15815815815816	0.211055785385155\\
5.18618618618619	0.213594348292612\\
5.21421421421421	0.216103092760253\\
5.24224224224224	0.218583045496552\\
5.27027027027027	0.221035175609107\\
5.2982982982983	0.223460399029783\\
5.32632632632633	0.225859582512124\\
5.35435435435435	0.228233547250513\\
5.38238238238238	0.230583072163924\\
5.41041041041041	0.232908896881494\\
5.43843843843844	0.235211724462356\\
5.46646646646647	0.237492223878081\\
5.49449449449449	0.239751032282569\\
5.52252252252252	0.24198875709123\\
5.55055055055055	0.244205977888666\\
5.57857857857858	0.246403248181839\\
5.60660660660661	0.248581097013755\\
5.63463463463463	0.250740030450989\\
5.66266266266266	0.252880532956896\\
5.69069069069069	0.255003068661075\\
5.71871871871872	0.257108082534498\\
5.74674674674675	0.25919600147874\\
5.77477477477477	0.261267235336864\\
5.8028028028028	0.263322177832725\\
5.83083083083083	0.265361207444816\\
5.85885885885886	0.267384688220121\\
5.88688688688689	0.269392970532961\\
5.91491491491491	0.271386391793276\\
5.94294294294294	0.273365277108441\\
5.97097097097097	0.275329939902249\\
5.998998998999	0.27728068249443\\
6.02702702702703	0.279217796643727\\
6.05505505505506	0.28114156405729\\
6.08308308308308	0.283052256868904\\
6.11111111111111	0.28495013808837\\
6.13913913913914	0.286835462024106\\
6.16716716716717	0.28870847468092\\
6.1951951951952	0.290569414134703\\
6.22322322322322	0.292418510885658\\
6.25125125125125	0.294255988191552\\
6.27927927927928	0.296082062382353\\
6.30730730730731	0.29789694315751\\
6.33533533533534	0.299700833867025\\
6.36336336336336	0.301493931777392\\
6.39139139139139	0.303276428323389\\
6.41941941941942	0.305048509346617\\
6.44744744744745	0.306810355321645\\
6.47547547547548	0.308562141570536\\
6.5035035035035	0.310304038466463\\
6.53153153153153	0.312036211627108\\
6.55955955955956	0.313758822098437\\
6.58758758758759	0.315472026529452\\
6.61561561561562	0.317175977338449\\
6.64364364364364	0.318870822871271\\
6.67167167167167	0.320556707552038\\
6.6996996996997	0.322233772026775\\
6.72772772772773	0.323902153300352\\
6.75575575575576	0.3255619848671\\
6.78378378378378	0.327213396835471\\
6.81181181181181	0.328856516047057\\
6.83983983983984	0.330491466190292\\
6.86786786786787	0.332118367909109\\
6.8958958958959	0.333737338906829\\
6.92392392392392	0.335348494045541\\
6.95195195195195	0.336951945441206\\
6.97997997997998	0.338547802554698\\
7.00800800800801	0.340136172279009\\
7.03603603603604	0.341717159022805\\
7.06406406406406	0.343290864790512\\
7.09209209209209	0.344857389259124\\
7.12012012012012	0.346416829851871\\
7.14814814814815	0.347969281808928\\
7.17617617617618	0.349514838255291\\
7.2042042042042	0.351053590265964\\
7.23223223223223	0.352585626928586\\
7.26026026026026	0.354111035403622\\
7.28828828828829	0.355629900982224\\
7.31631631631632	0.357142307141884\\
7.34434434434434	0.35864833559997\\
7.37237237237237	0.360148066365252\\
7.4004004004004	0.361641577787498\\
7.42842842842843	0.363128946605241\\
7.45645645645646	0.364610247991788\\
7.48448448448448	0.366085555599554\\
7.51251251251251	0.367554941602796\\
7.54054054054054	0.369018476738816\\
7.56856856856857	0.370476230347693\\
7.5965965965966	0.371928270410624\\
7.62462462462462	0.373374663586912\\
7.65265265265265	0.374815475249676\\
7.68068068068068	0.376250769520331\\
7.70870870870871	0.377680609301884\\
7.73673673673674	0.379105056311102\\
7.76476476476476	0.380524171109596\\
7.79279279279279	0.381938013133867\\
7.82082082082082	0.383346640724354\\
7.84884884884885	0.384750111153523\\
7.87687687687688	0.386148480653037\\
7.9049049049049	0.38754180444005\\
7.93293293293293	0.388930136742643\\
7.96096096096096	0.390313530824454\\
7.98898898898899	0.391692039008522\\
8.01701701701702	0.393065712700376\\
8.04504504504505	0.394434602410407\\
8.07307307307307	0.395798757775535\\
8.1011011011011	0.397158227580215\\
8.12912912912913	0.398513059776789\\
8.15715715715716	0.399863301505227\\
8.18518518518519	0.401208999112259\\
8.21321321321321	0.402550198169945\\
8.24124124124124	0.403886943493675\\
8.26926926926927	0.405219279159647\\
8.2972972972973	0.406547248521824\\
8.32532532532533	0.40787089422839\\
8.35335335335335	0.409190258237739\\
8.38138138138138	0.410505381833988\\
8.40940940940941	0.411816305642054\\
8.43743743743744	0.413123069642293\\
8.46546546546547	0.41442571318473\\
8.49349349349349	0.415724275002883\\
8.52152152152152	0.417018793227204\\
8.54954954954955	0.418309305398141\\
8.57757757757758	0.419595848478838\\
8.60560560560561	0.420878458867494\\
8.63363363363363	0.422157172409374\\
8.66166166166166	0.423432024408498\\
8.68968968968969	0.424703049639016\\
8.71771771771772	0.425970282356271\\
8.74574574574575	0.427233756307577\\
8.77377377377377	0.428493504742702\\
8.8018018018018	0.429749560424074\\
8.82982982982983	0.43100195563673\\
8.85785785785786	0.432250722197993\\
8.88588588588589	0.433495891466906\\
8.91391391391391	0.434737494353419\\
8.94194194194194	0.435975561327341\\
8.96996996996997	0.437210122427067\\
8.997997997998	0.438441207268081\\
9.02602602602603	0.439668845051242\\
9.05405405405405	0.440893064570871\\
9.08208208208208	0.442113894222626\\
9.11011011011011	0.44333136201119\\
9.13813813813814	0.444545495557764\\
9.16616616616617	0.44575632210738\\
9.19419419419419	0.446963868536037\\
9.22222222222222	0.448168161357655\\
9.25025025025025	0.449369226730872\\
9.27827827827828	0.450567090465674\\
9.30630630630631	0.451761778029864\\
9.33433433433433	0.45295331455538\\
9.36236236236236	0.454141724844465\\
9.39039039039039	0.455327033375691\\
9.41841841841842	0.456509264309837\\
9.44644644644645	0.457688441495642\\
9.47447447447447	0.458864588475414\\
9.5025025025025	0.460037728490514\\
9.53053053053053	0.461207884486717\\
9.55855855855856	0.462375079119446\\
9.58658658658659	0.463539334758891\\
9.61461461461461	0.46470067349501\\
9.64264264264264	0.465859117142422\\
9.67067067067067	0.467014687245185\\
9.6986986986987	0.468167405081472\\
9.72672672672673	0.469317291668142\\
9.75475475475475	0.470464367765213\\
9.78278278278278	0.47160865388023\\
9.81081081081081	0.472750170272548\\
9.83883883883884	0.473888936957516\\
9.86686686686687	0.475024973710568\\
9.89489489489489	0.476158300071233\\
9.92292292292292	0.477288935347058\\
9.95095095095095	0.478416898617439\\
9.97897897897898	0.479542208737385\\
10.007007007007	0.480664884341192\\
10.035035035035	0.48178494384604\\
10.0630630630631	0.482902405455526\\
10.0910910910911	0.484017287163108\\
10.1191191191191	0.485129606755489\\
10.1471471471471	0.486239381815929\\
10.1751751751752	0.487346629727484\\
10.2032032032032	0.488451367676188\\
10.2312312312312	0.48955361265416\\
10.2592592592593	0.490653381462658\\
10.2872872872873	0.491750690715063\\
10.3153153153153	0.492845556839811\\
10.3433433433433	0.493937996083257\\
10.3713713713714	0.495028024512497\\
10.3993993993994	0.496115658018114\\
10.4274274274274	0.49720091231689\\
10.4554554554555	0.498283802954449\\
10.4834834834835	0.499364345307862\\
10.5115115115115	0.500442554588187\\
10.5395395395395	0.501518445842973\\
10.5675675675676	0.50259203395871\\
10.5955955955956	0.503663333663226\\
10.6236236236236	0.504732359528052\\
10.6516516516517	0.505799125970731\\
10.6796796796797	0.506863647257087\\
10.7077077077077	0.50792593750345\\
10.7357357357357	0.508986010678839\\
10.7637637637638	0.51004388060711\\
10.7917917917918	0.511099560969055\\
10.8198198198198	0.512153065304468\\
10.8478478478478	0.513204407014168\\
10.8758758758759	0.514253599361992\\
10.9039039039039	0.515300655476747\\
10.9319319319319	0.516345588354124\\
10.95995995996	0.517388410858581\\
10.987987987988	0.518429135725191\\
11.016016016016	0.519467775561459\\
11.044044044044	0.520504342849098\\
11.0720720720721	0.521538849945785\\
11.1001001001001	0.522571309086878\\
11.1281281281281	0.523601732387102\\
11.1561561561562	0.524630131842211\\
11.1841841841842	0.525656519330614\\
11.2122122122122	0.52668090661498\\
11.2402402402402	0.527703305343806\\
11.2682682682683	0.528723727052966\\
11.2962962962963	0.529742183167228\\
11.3243243243243	0.530758685001745\\
11.3523523523524	0.531773243763526\\
11.3803803803804	0.532785870552874\\
11.4084084084084	0.533796576364803\\
11.4364364364364	0.534805372090433\\
11.4644644644645	0.535812268518356\\
11.4924924924925	0.536817276335988\\
11.5205205205205	0.537820406130884\\
11.5485485485485	0.538821668392048\\
11.5765765765766	0.539821073511206\\
11.6046046046046	0.540818631784069\\
11.6326326326326	0.541814353411566\\
11.6606606606607	0.542808248501066\\
11.6886886886887	0.54380032706757\\
11.7167167167167	0.544790599034892\\
11.7447447447447	0.545779074236818\\
11.7727727727728	0.54676576241824\\
11.8008008008008	0.547750673236282\\
11.8288288288288	0.548733816261401\\
11.8568568568569	0.549715200978472\\
11.8848848848849	0.550694836787856\\
11.9129129129129	0.551672733006449\\
11.9409409409409	0.552648898868717\\
11.968968968969	0.553623343527715\\
11.996996996997	0.554596076056086\\
12.025025025025	0.555567105447046\\
12.0530530530531	0.556536440615356\\
12.0810810810811	0.557504090398278\\
12.1091091091091	0.55847006355651\\
12.1371371371371	0.559434368775116\\
12.1651651651652	0.560397014664438\\
12.1931931931932	0.561358009760986\\
12.2212212212212	0.562317362528329\\
12.2492492492493	0.563275081357961\\
12.2772772772773	0.564231174570156\\
12.3053053053053	0.565185650414817\\
12.3333333333333	0.566138517072298\\
12.3613613613614	0.56708978265423\\
12.3893893893894	0.568039455204322\\
12.4174174174174	0.568987542699156\\
12.4454454454454	0.569934053048969\\
12.4734734734735	0.570878994098422\\
12.5015015015015	0.571822373627359\\
12.5295295295295	0.572764199351555\\
12.5575575575576	0.573704478923452\\
12.5855855855856	0.574643219932882\\
12.6136136136136	0.575580429907786\\
12.6416416416416	0.576516116314915\\
12.6696696696697	0.577450286560526\\
12.6976976976977	0.578382947991065\\
12.7257257257257	0.57931410789384\\
12.7537537537538	0.580243773497688\\
12.7817817817818	0.581171951973629\\
12.8098098098098	0.582098650435507\\
12.8378378378378	0.583023875940634\\
12.8658658658659	0.58394763549041\\
12.8938938938939	0.584869936030945\\
12.9219219219219	0.585790784453668\\
12.94994994995	0.586710187595924\\
12.977977977978	0.587628152241573\\
13.006006006006	0.588544685121569\\
13.034034034034	0.589459792914536\\
13.0620620620621	0.590373482247338\\
13.0900900900901	0.59128575969564\\
13.1181181181181	0.592196631784453\\
13.1461461461461	0.593106104988687\\
13.1741741741742	0.594014185733685\\
13.2022022022022	0.59492088039575\\
13.2302302302302	0.59582619530267\\
13.2582582582583	0.596730136734233\\
13.2862862862863	0.597632710922735\\
13.3143143143143	0.598533924053481\\
13.3423423423423	0.59943378226528\\
13.3703703703704	0.600332291650933\\
13.3983983983984	0.601229458257712\\
13.4264264264264	0.602125288087838\\
13.4544544544545	0.603019787098951\\
13.4824824824825	0.603912961204564\\
13.5105105105105	0.604804816274528\\
13.5385385385385	0.60569535813548\\
13.5665665665666	0.606584592571282\\
13.5945945945946	0.607472525323466\\
13.6226226226226	0.608359162091663\\
13.6506506506507	0.60924450853403\\
13.6786786786787	0.610128570267671\\
13.7067067067067	0.611011352869055\\
13.7347347347347	0.611892861874425\\
13.7627627627628	0.612773102780202\\
13.7907907907908	0.613652081043385\\
13.8188188188188	0.614529802081948\\
13.8468468468468	0.615406271275227\\
13.8748748748749	0.616281493964308\\
13.9029029029029	0.617155475452402\\
13.9309309309309	0.618028221005224\\
13.958958958959	0.61889973585136\\
13.986986986987	0.619770025182635\\
14.015015015015	0.620639094154473\\
14.043043043043	0.621506947886254\\
14.0710710710711	0.622373591461662\\
14.0990990990991	0.623239029929038\\
14.1271271271271	0.624103268301721\\
14.1551551551552	0.624966311558384\\
14.1831831831832	0.625828164643372\\
14.2112112112112	0.626688832467033\\
14.2392392392392	0.62754831990604\\
14.2672672672673	0.628406631803719\\
14.2952952952953	0.629263772970363\\
14.3233233233233	0.630119748183547\\
14.3513513513514	0.630974562188444\\
14.3793793793794	0.631828219698123\\
14.4074074074074	0.63268072539386\\
14.4354354354354	0.633532083925433\\
14.4634634634635	0.63438229991142\\
14.4914914914915	0.63523137793949\\
14.5195195195195	0.636079322566694\\
14.5475475475475	0.636926138319747\\
14.5755755755756	0.637771829695313\\
14.6036036036036	0.63861640116028\\
14.6316316316316	0.639459857152041\\
14.6596596596597	0.640302202078761\\
14.6876876876877	0.641143440319646\\
14.7157157157157	0.641983576225212\\
14.7437437437437	0.642822614117544\\
14.7717717717718	0.643660558290559\\
14.7997997997998	0.644497413010258\\
14.8278278278278	0.645333182514983\\
14.8558558558559	0.646167871015665\\
14.8838838838839	0.647001482696074\\
14.9119119119119	0.647834021713062\\
14.9399399399399	0.648665492196803\\
14.967967967968	0.649495898251037\\
14.995995995996	0.650325243953303\\
15.024024024024	0.651153533355171\\
15.0520520520521	0.651980770482477\\
15.0800800800801	0.65280695933555\\
15.1081081081081	0.653632103889434\\
15.1361361361361	0.654456208094116\\
15.1641641641642	0.655279275874744\\
15.1921921921922	0.656101311131848\\
15.2202202202202	0.656922317741551\\
15.2482482482482	0.657742299555785\\
15.2762762762763	0.658561260402503\\
15.3043043043043	0.659379204085885\\
15.3323323323323	0.660196134386544\\
15.3603603603604	0.661012055061734\\
15.3883883883884	0.661826969845547\\
15.4164164164164	0.662640882449113\\
15.4444444444444	0.663453796560801\\
15.4724724724725	0.664265715846408\\
15.5005005005005	0.665076643949359\\
15.5285285285285	0.66588658449089\\
15.5565565565566	0.666695541070243\\
15.5845845845846	0.66750351726485\\
15.6126126126126	0.668310516630516\\
15.6406406406406	0.669116542701604\\
15.6686686686687	0.669921598991217\\
15.6966966966967	0.670725688991371\\
15.7247247247247	0.671528816173179\\
15.7527527527528	0.672330983987018\\
15.7807807807808	0.673132195862709\\
15.8088088088088	0.673932455209685\\
15.8368368368368	0.674731765417158\\
15.8648648648649	0.675530129854292\\
15.8928928928929	0.676327551870362\\
15.9209209209209	0.677124034794925\\
15.9489489489489	0.677919581937976\\
15.976976976977	0.678714196590112\\
16.005005005005	0.67950788202269\\
16.033033033033	0.680300641487984\\
16.0610610610611	0.681092478219341\\
16.0890890890891	0.681883395431334\\
16.1171171171171	0.682673396319914\\
16.1451451451451	0.683462484062561\\
16.1731731731732	0.684250661818436\\
16.2012012012012	0.685037932728524\\
16.2292292292292	0.68582429991578\\
16.2572572572573	0.68660976648528\\
16.2852852852853	0.687394335524357\\
16.3133133133133	0.688178010102745\\
16.3413413413413	0.688960793272719\\
16.3693693693694	0.689742688069236\\
16.3973973973974	0.690523697510069\\
16.4254254254254	0.691303824595943\\
16.4534534534535	0.692083072310672\\
16.4814814814815	0.692861443621291\\
16.5095095095095	0.693638941478188\\
16.5375375375375	0.694415568815232\\
16.5655655655656	0.695191328549906\\
16.5935935935936	0.695966223583433\\
16.6216216216216	0.696740256800901\\
16.6496496496497	0.697513431071393\\
16.6776776776777	0.698285749248105\\
16.7057057057057	0.699057214168475\\
16.7337337337337	0.699827828654298\\
16.7617617617618	0.700597595511854\\
16.7897897897898	0.701366517532019\\
16.8178178178178	0.702134597490391\\
16.8458458458458	0.7029018381474\\
16.8738738738739	0.70366824224843\\
16.9019019019019	0.704433812523928\\
16.9299299299299	0.705198551689521\\
16.957957957958	0.705962462446126\\
16.985985985986	0.706725547480066\\
17.014014014014	0.707487809463174\\
17.042042042042	0.708249251052906\\
17.0700700700701	0.709009874892446\\
17.0980980980981	0.70976968361082\\
17.1261261261261	0.710528679822993\\
17.1541541541542	0.711286866129979\\
17.1821821821822	0.712044245118944\\
17.2102102102102	0.712800819363311\\
17.2382382382382	0.713556591422859\\
17.2662662662663	0.714311563843825\\
17.2942942942943	0.715065739159003\\
17.3223223223223	0.715819119887847\\
17.3503503503504	0.716571708536566\\
17.3783783783784	0.717323507598221\\
17.4064064064064	0.718074519552822\\
17.4344344344344	0.718824746867424\\
17.4624624624625	0.719574191996223\\
17.4904904904905	0.720322857380646\\
17.5185185185185	0.721070745449446\\
17.5465465465465	0.721817858618794\\
17.5745745745746	0.722564199292369\\
17.6026026026026	0.72330976986145\\
17.6306306306306	0.724054572705003\\
17.6586586586587	0.724798610189772\\
17.6866866866867	0.725541884670366\\
17.7147147147147	0.726284398489344\\
17.7427427427427	0.727026153977303\\
17.7707707707708	0.727767153452964\\
17.7987987987988	0.728507399223255\\
17.8268268268268	0.729246893583395\\
17.8548548548549	0.729985638816979\\
17.8828828828829	0.730723637196056\\
17.9109109109109	0.731460890981216\\
17.9389389389389	0.732197402421667\\
17.966966966967	0.732933173755314\\
17.994994994995	0.733668207208843\\
18.023023023023	0.734402504997797\\
18.0510510510511	0.735136069326652\\
18.0790790790791	0.735868902388898\\
18.1071071071071	0.736601006367112\\
18.1351351351351	0.737332383433037\\
18.1631631631632	0.738063035747655\\
18.1911911911912	0.738792965461262\\
18.2192192192192	0.739522174713542\\
18.2472472472472	0.740250665633641\\
18.2752752752753	0.740978440340236\\
18.3033033033033	0.741705500941614\\
18.3313313313313	0.742431849535734\\
18.3593593593594	0.743157488210307\\
18.3873873873874	0.743882419042857\\
18.4154154154154	0.744606644100799\\
18.4434434434434	0.745330165441501\\
18.4714714714715	0.746052985112356\\
18.4994994994995	0.746775105150846\\
18.5275275275275	0.747496527584616\\
18.5555555555556	0.748217254431532\\
18.5835835835836	0.748937287699752\\
18.6116116116116	0.74965662938779\\
18.6396396396396	0.75037528148458\\
18.6676676676677	0.751093245969543\\
18.6956956956957	0.751810524812645\\
18.7237237237237	0.752527119974467\\
18.7517517517518	0.753243033406262\\
18.7797797797798	0.753958267050019\\
18.8078078078078	0.754672822838526\\
18.8358358358358	0.755386702695428\\
18.8638638638639	0.756099908535289\\
18.8918918918919	0.756812442263651\\
18.9199199199199	0.757524305777098\\
18.9479479479479	0.758235500963306\\
18.975975975976	0.758946029701109\\
19.004004004004	0.759655893860554\\
19.032032032032	0.760365095302959\\
19.0600600600601	0.76107363588097\\
19.0880880880881	0.761781517438616\\
19.1161161161161	0.762488741811367\\
19.1441441441441	0.763195310826189\\
19.1721721721722	0.763901226301597\\
19.2002002002002	0.764606490047714\\
19.2282282282282	0.76531110386632\\
19.2562562562563	0.766015069550911\\
19.2842842842843	0.766718388886746\\
19.3123123123123	0.767421063650907\\
19.3403403403403	0.768123095612344\\
19.3683683683684	0.768824486531933\\
19.3963963963964	0.769525238162526\\
19.4244244244244	0.770225352248998\\
19.4524524524525	0.770924830528304\\
19.4804804804805	0.771623674729525\\
19.5085085085085	0.772321886573918\\
19.5365365365365	0.77301946777497\\
19.5645645645646	0.773716420038439\\
19.5925925925926	0.774412745062412\\
19.6206206206206	0.775108444537344\\
19.6486486486486	0.775803520146114\\
19.6766766766767	0.776497973564067\\
19.7047047047047	0.777191806459064\\
19.7327327327327	0.777885020491528\\
19.7607607607608	0.778577617314489\\
19.7887887887888	0.779269598573631\\
19.8168168168168	0.779960965907338\\
19.8448448448448	0.780651720946742\\
19.8728728728729	0.781341865315759\\
19.9009009009009	0.782031400631146\\
19.9289289289289	0.782720328502533\\
19.956956956957	0.783408650532475\\
19.984984984985	0.784096368316494\\
20.013013013013	0.784783483443119\\
20.041041041041	0.785469997493932\\
20.0690690690691	0.786155912043609\\
20.0970970970971	0.786841228659962\\
20.1251251251251	0.787525948903984\\
20.1531531531532	0.788210074329884\\
20.1811811811812	0.788893606485135\\
20.2092092092092	0.789576546910509\\
20.2372372372372	0.790258897140124\\
20.2652652652653	0.790940658701477\\
20.2932932932933	0.791621833115489\\
20.3213213213213	0.792302421896543\\
20.3493493493494	0.792982426552524\\
20.3773773773774	0.793661848584855\\
20.4054054054054	0.794340689488541\\
20.4334334334334	0.795018950752202\\
20.4614614614615	0.795696633858114\\
20.4894894894895	0.796373740282246\\
20.5175175175175	0.797050271494299\\
20.5455455455455	0.79772622895774\\
20.5735735735736	0.798401614129842\\
20.6016016016016	0.79907642846172\\
20.6296296296296	0.799750673398366\\
20.6576576576577	0.800424350378686\\
20.6856856856857	0.801097460835535\\
20.7137137137137	0.801770006195756\\
20.7417417417417	0.80244198788021\\
20.7697697697698	0.803113407303816\\
20.7977977977978	0.803784265875581\\
20.8258258258258	0.804454564998638\\
20.8538538538539	0.805124306070279\\
20.8818818818819	0.80579349048199\\
20.9099099099099	0.806462119619484\\
20.9379379379379	0.807130194862732\\
20.965965965966	0.807797717586001\\
20.993993993994	0.808464689157884\\
21.022022022022	0.809131110941332\\
21.0500500500501	0.809796984293691\\
21.0780780780781	0.810462310566726\\
21.1061061061061	0.811127091106664\\
21.1341341341341	0.811791327254214\\
21.1621621621622	0.81245502034461\\
21.1901901901902	0.813118171707631\\
21.2182182182182	0.813780782667643\\
21.2462462462462	0.814442854543622\\
21.2742742742743	0.815104388649187\\
21.3023023023023	0.815765386292631\\
21.3303303303303	0.816425848776952\\
21.3583583583584	0.817085777399881\\
21.3863863863864	0.817745173453912\\
21.4144144144144	0.818404038226333\\
21.4424424424424	0.819062372999255\\
21.4704704704705	0.819720179049639\\
21.4984984984985	0.820377457649327\\
21.5265265265265	0.82103421006507\\
21.5545545545546	0.821690437558558\\
21.5825825825826	0.822346141386446\\
21.6106106106106	0.823001322800381\\
21.6386386386386	0.823655983047036\\
21.6666666666667	0.824310123368129\\
21.6946946946947	0.824963745000458\\
21.7227227227227	0.825616849175925\\
21.7507507507508	0.826269437121562\\
21.7787787787788	0.82692151005956\\
21.8068068068068	0.827573069207297\\
21.8348348348348	0.828224115777358\\
21.8628628628629	0.828874650977569\\
21.8908908908909	0.82952467601102\\
21.9189189189189	0.83017419207609\\
21.9469469469469	0.830823200366474\\
21.974974974975	0.831471702071208\\
22.003003003003	0.832119698374694\\
22.031031031031	0.832767190456729\\
22.0590590590591	0.833414179492523\\
22.0870870870871	0.83406066665273\\
22.1151151151151	0.83470665310347\\
22.1431431431431	0.835352140006355\\
22.1711711711712	0.835997128518513\\
22.1991991991992	0.836641619792609\\
22.2272272272272	0.837285614976873\\
22.2552552552553	0.837929115215125\\
22.2832832832833	0.838572121646793\\
22.3113113113113	0.839214635406941\\
22.3393393393393	0.839856657626291\\
22.3673673673674	0.840498189431248\\
22.3953953953954	0.84113923194392\\
22.4234234234234	0.841779786282142\\
22.4514514514515	0.8424198535595\\
22.4794794794795	0.843059434885353\\
22.5075075075075	0.843698531364854\\
22.5355355355355	0.844337144098975\\
22.5635635635636	0.844975274184525\\
22.5915915915916	0.845612922714178\\
22.6196196196196	0.846250090776489\\
22.6476476476476	0.846886779455918\\
22.6756756756757	0.847522989832853\\
22.7037037037037	0.848158722983628\\
22.7317317317317	0.848793979980549\\
22.7597597597598	0.849428761891909\\
22.7877877877878	0.850063069782015\\
22.8158158158158	0.850696904711204\\
22.8438438438438	0.851330267735868\\
22.8718718718719	0.85196315990847\\
22.8998998998999	0.852595582277569\\
22.9279279279279	0.853227535887836\\
22.955955955956	0.853859021780076\\
22.983983983984	0.854490040991251\\
23.012012012012	0.855120594554494\\
23.04004004004	0.855750683499132\\
23.0680680680681	0.856380308850708\\
23.0960960960961	0.857009471630993\\
23.1241241241241	0.857638172858016\\
23.1521521521522	0.858266413546072\\
23.1801801801802	0.858894194705749\\
23.2082082082082	0.859521517343946\\
23.2362362362362	0.860148382463886\\
23.2642642642643	0.860774791065143\\
23.2922922922923	0.861400744143655\\
23.3203203203203	0.862026242691743\\
23.3483483483483	0.862651287698131\\
23.3763763763764	0.863275880147966\\
23.4044044044044	0.86390002102283\\
23.4324324324324	0.864523711300764\\
23.4604604604605	0.865146951956284\\
23.4884884884885	0.865769743960396\\
23.5165165165165	0.866392088280618\\
23.5445445445445	0.867013985880996\\
23.5725725725726	0.867635437722119\\
23.6006006006006	0.86825644476114\\
23.6286286286286	0.86887700795179\\
23.6566566566567	0.869497128244399\\
23.6846846846847	0.870116806585907\\
23.7127127127127	0.870736043919889\\
23.7407407407407	0.871354841186562\\
23.7687687687688	0.871973199322812\\
23.7967967967968	0.872591119262201\\
23.8248248248248	0.873208601934992\\
23.8528528528529	0.873825648268156\\
23.8808808808809	0.874442259185399\\
23.9089089089089	0.875058435607169\\
23.9369369369369	0.875674178450676\\
23.964964964965	0.876289488629909\\
23.992992992993	0.876904367055649\\
24.021021021021	0.877518814635487\\
24.049049049049	0.878132832273837\\
24.0770770770771	0.878746420871956\\
24.1051051051051	0.879359581327953\\
24.1331331331331	0.879972314536812\\
24.1611611611612	0.880584621390401\\
24.1891891891892	0.881196502777489\\
24.2172172172172	0.881807959583762\\
24.2452452452452	0.882418992691838\\
24.2732732732733	0.883029602981282\\
24.3013013013013	0.883639791328616\\
24.3293293293293	0.884249558607344\\
24.3573573573574	0.884858905687954\\
24.3853853853854	0.885467833437943\\
24.4134134134134	0.886076342721825\\
24.4414414414414	0.886684434401149\\
24.4694694694695	0.887292109334512\\
24.4974974974975	0.88789936837757\\
24.5255255255255	0.888506212383059\\
24.5535535535536	0.889112642200802\\
24.5815815815816	0.889718658677726\\
24.6096096096096	0.890324262657877\\
24.6376376376376	0.890929454982432\\
24.6656656656657	0.891534236489713\\
24.6936936936937	0.892138608015198\\
24.7217217217217	0.892742570391542\\
24.7497497497498	0.89334612444858\\
24.7777777777778	0.89394927101335\\
24.8058058058058	0.894552010910098\\
24.8338338338338	0.895154344960299\\
24.8618618618619	0.895756273982662\\
24.8898898898899	0.896357798793151\\
24.9179179179179	0.896958920204992\\
24.9459459459459	0.897559639028686\\
24.973973973974	0.898159956072028\\
25.002002002002	0.898759872140111\\
25.03003003003	0.899359388035344\\
25.0580580580581	0.899958504557466\\
25.0860860860861	0.900557222503551\\
25.1141141141141	0.90115554266803\\
25.1421421421421	0.901753465842695\\
25.1701701701702	0.902350992816717\\
25.1981981981982	0.902948124376653\\
25.2262262262262	0.903544861306464\\
25.2542542542543	0.904141204387521\\
25.2822822822823	0.904737154398623\\
25.3103103103103	0.905332712116004\\
25.3383383383383	0.905927878313346\\
25.3663663663664	0.906522653761792\\
25.3943943943944	0.907117039229957\\
25.4224224224224	0.907711035483939\\
25.4504504504505	0.908304643287331\\
25.4784784784785	0.908897863401233\\
25.5065065065065	0.909490696584263\\
25.5345345345345	0.910083143592567\\
25.5625625625626	0.910675205179832\\
25.5905905905906	0.911266882097298\\
25.6186186186186	0.911858175093765\\
25.6466466466466	0.912449084915609\\
25.6746746746747	0.91303961230679\\
25.7027027027027	0.913629758008864\\
25.7307307307307	0.914219522760992\\
25.7587587587588	0.914808907299955\\
25.7867867867868	0.915397912360159\\
25.8148148148148	0.915986538673653\\
25.8428428428428	0.916574786970131\\
25.8708708708709	0.917162657976949\\
25.8988988988989	0.917750152419133\\
25.9269269269269	0.918337271019392\\
25.954954954955	0.918924014498122\\
25.982982982983	0.919510383573424\\
26.011011011011	0.92009637896111\\
26.039039039039	0.920682001374713\\
26.0670670670671	0.9212672515255\\
26.0950950950951	0.921852130122478\\
26.1231231231231	0.922436637872407\\
26.1511511511512	0.923020775479811\\
26.1791791791792	0.923604543646984\\
26.2072072072072	0.924187943074002\\
26.2352352352352	0.924770974458733\\
26.2632632632633	0.925353638496848\\
26.2912912912913	0.925935935881828\\
26.3193193193193	0.926517867304975\\
26.3473473473473	0.927099433455419\\
26.3753753753754	0.927680635020135\\
26.4034034034034	0.928261472683942\\
26.4314314314314	0.928841947129523\\
26.4594594594595	0.929422059037423\\
26.4874874874875	0.930001809086071\\
26.5155155155155	0.930581197951778\\
26.5435435435435	0.931160226308752\\
26.5715715715716	0.931738894829108\\
26.5995995995996	0.932317204182873\\
26.6276276276276	0.932895155037998\\
26.6556556556557	0.933472748060366\\
26.6836836836837	0.934049983913801\\
26.7117117117117	0.934626863260078\\
26.7397397397397	0.935203386758931\\
26.7677677677678	0.93577955506806\\
26.7957957957958	0.936355368843142\\
26.8238238238238	0.936930828737841\\
26.8518518518519	0.937505935403813\\
26.8798798798799	0.938080689490718\\
26.9079079079079	0.938655091646224\\
26.9359359359359	0.939229142516022\\
26.963963963964	0.93980284274383\\
26.991991991992	0.9403761929714\\
27.02002002002	0.940949193838533\\
27.048048048048	0.94152184598308\\
27.0760760760761	0.942094150040954\\
27.1041041041041	0.942666106646138\\
27.1321321321321	0.943237716430694\\
27.1601601601602	0.943808980024767\\
27.1881881881882	0.944379898056599\\
27.2162162162162	0.944950471152532\\
27.2442442442442	0.945520699937021\\
27.2722722722723	0.946090585032636\\
27.3003003003003	0.946660127060074\\
27.3283283283283	0.947229326638168\\
27.3563563563564	0.94779818438389\\
27.3843843843844	0.948366700912363\\
27.4124124124124	0.948934876836867\\
27.4404404404404	0.949502712768848\\
27.4684684684685	0.950070209317923\\
27.4964964964965	0.950637367091892\\
27.5245245245245	0.951204186696739\\
27.5525525525526	0.951770668736648\\
27.5805805805806	0.952336813814003\\
27.6086086086086	0.9529026225294\\
27.6366366366366	0.953468095481652\\
27.6646646646647	0.954033233267799\\
27.6926926926927	0.954598036483111\\
27.7207207207207	0.955162505721101\\
27.7487487487487	0.955726641573527\\
27.7767767767768	0.956290444630403\\
27.8048048048048	0.956853915480004\\
27.8328328328328	0.957417054708875\\
27.8608608608609	0.957979862901835\\
27.8888888888889	0.958542340641987\\
27.9169169169169	0.959104488510725\\
27.9449449449449	0.95966630708774\\
27.972972972973	0.960227796951027\\
28.001001001001	0.96078895867689\\
28.029029029029	0.961349792839955\\
28.0570570570571	0.961910300013169\\
28.0850850850851	0.962470480767814\\
28.1131131131131	0.963030335673507\\
28.1411411411411	0.963589865298213\\
28.1691691691692	0.964149070208248\\
28.1971971971972	0.964707950968287\\
28.2252252252252	0.965266508141369\\
28.2532532532533	0.965824742288907\\
28.2812812812813	0.96638265397069\\
28.3093093093093	0.966940243744894\\
28.3373373373373	0.967497512168086\\
28.3653653653654	0.968054459795231\\
28.3933933933934	0.968611087179699\\
28.4214214214214	0.969167394873269\\
28.4494494494495	0.96972338342614\\
28.4774774774775	0.970279053386932\\
28.5055055055055	0.970834405302698\\
28.5335335335335	0.971389439718923\\
28.5615615615616	0.971944157179539\\
28.5895895895896	0.972498558226924\\
28.6176176176176	0.973052643401911\\
28.6456456456456	0.973606413243795\\
28.6736736736737	0.974159868290338\\
28.7017017017017	0.974713009077775\\
28.7297297297297	0.975265836140819\\
28.7577577577578	0.975818350012671\\
28.7857857857858	0.976370551225022\\
28.8138138138138	0.97692244030806\\
28.8418418418418	0.977474017790476\\
28.8698698698699	0.978025284199471\\
28.8978978978979	0.978576240060762\\
28.9259259259259	0.979126885898584\\
28.953953953954	0.979677222235701\\
28.981981981982	0.98022724959341\\
29.01001001001	0.980776968491543\\
29.038038038038	0.98132637944848\\
29.0660660660661	0.981875482981149\\
29.0940940940941	0.982424279605032\\
29.1221221221221	0.982972769834173\\
29.1501501501502	0.983520954181184\\
29.1781781781782	0.984068833157248\\
29.2062062062062	0.984616407272126\\
29.2342342342342	0.985163677034162\\
29.2622622622623	0.985710642950289\\
29.2902902902903	0.986257305526034\\
29.3183183183183	0.986803665265525\\
29.3463463463463	0.987349722671493\\
29.3743743743744	0.987895478245283\\
29.4024024024024	0.988440932486852\\
29.4304304304304	0.988986085894781\\
29.4584584584585	0.989530938966276\\
29.4864864864865	0.990075492197176\\
29.5145145145145	0.990619746081957\\
29.5425425425425	0.991163701113738\\
29.5705705705706	0.991707357784283\\
29.5985985985986	0.992250716584012\\
29.6266266266266	0.992793778002001\\
29.6546546546547	0.993336542525989\\
29.6826826826827	0.993879010642386\\
29.7107107107107	0.994421182836271\\
29.7387387387387	0.994963059591405\\
29.7667667667668	0.99550464139023\\
29.7947947947948	0.996045928713878\\
29.8228228228228	0.996586922042173\\
29.8508508508509	0.997127621853639\\
29.8788788788789	0.997668028625502\\
29.9069069069069	0.998208142833696\\
29.9349349349349	0.998747964952869\\
29.962962962963	0.999287495456387\\
29.990990990991	0.999826734816338\\
30.019019019019	1\\
30.047047047047	1\\
30.0750750750751	1\\
30.1031031031031	1\\
30.1311311311311	1\\
30.1591591591592	1\\
30.1871871871872	1\\
30.2152152152152	1\\
30.2432432432432	1\\
30.2712712712713	1\\
30.2992992992993	1\\
30.3273273273273	1\\
30.3553553553554	1\\
30.3833833833834	1\\
30.4114114114114	1\\
30.4394394394394	1\\
30.4674674674675	1\\
30.4954954954955	1\\
30.5235235235235	1\\
30.5515515515516	1\\
30.5795795795796	1\\
30.6076076076076	1\\
30.6356356356356	1\\
30.6636636636637	1\\
30.6916916916917	1\\
30.7197197197197	1\\
30.7477477477477	1\\
30.7757757757758	1\\
30.8038038038038	1\\
30.8318318318318	1\\
30.8598598598599	1\\
30.8878878878879	1\\
30.9159159159159	1\\
30.9439439439439	1\\
30.971971971972	1\\
31	1\\
};
\addlegendentry{Square Root};

\addplot [color=mycolor3,solid]
  table[row sep=crcr]{%
3	0\\
3.02802802802803	0\\
3.05605605605606	0\\
3.08408408408408	0\\
3.11211211211211	0\\
3.14014014014014	0\\
3.16816816816817	0\\
3.1961961961962	0\\
3.22422422422422	0\\
3.25225225225225	0\\
3.28028028028028	0\\
3.30830830830831	0\\
3.33633633633634	0\\
3.36436436436436	0\\
3.39239239239239	0\\
3.42042042042042	0\\
3.44844844844845	0\\
3.47647647647648	0\\
3.5045045045045	0\\
3.53253253253253	0\\
3.56056056056056	0\\
3.58858858858859	0\\
3.61661661661662	0\\
3.64464464464464	0\\
3.67267267267267	0\\
3.7007007007007	0\\
3.72872872872873	0\\
3.75675675675676	0\\
3.78478478478478	0\\
3.81281281281281	0\\
3.84084084084084	0\\
3.86886886886887	0\\
3.8968968968969	0\\
3.92492492492493	0\\
3.95295295295295	0\\
3.98098098098098	0\\
4.00900900900901	0.0277246870709527\\
4.03703703703704	0.0523779262168572\\
4.06506506506507	0.0674945232759177\\
4.09309309309309	0.079300303700453\\
4.12112112112112	0.0892711111334547\\
4.14914914914915	0.0980373005544083\\
4.17717717717718	0.105936469081023\\
4.20520520520521	0.113174024862471\\
4.23323323323323	0.119885784126166\\
4.26126126126126	0.126166992879387\\
4.28928928928929	0.132087449062323\\
4.31731731731732	0.137700079727055\\
4.34534534534535	0.143046131214801\\
4.37337337337337	0.148158474615085\\
4.4014014014014	0.153063800126997\\
4.42942942942943	0.157784124790946\\
4.45745745745746	0.162337858919974\\
4.48548548548549	0.166740579325801\\
4.51351351351351	0.171005602103885\\
4.54154154154154	0.175144414961552\\
4.56956956956957	0.179167008965584\\
4.5975975975976	0.183082136871307\\
4.62562562562563	0.186897516937945\\
4.65365365365365	0.190619995643482\\
4.68168168168168	0.19425567898179\\
4.70970970970971	0.197810039441754\\
4.73773773773774	0.201288003948385\\
4.76576576576577	0.204694026743683\\
4.79379379379379	0.208032150239567\\
4.82182182182182	0.211306056179796\\
4.84984984984985	0.214519108929962\\
4.87787787787788	0.217674392324826\\
4.90590590590591	0.220774741205648\\
4.93393393393393	0.223822768552356\\
4.96196196196196	0.226820888938824\\
4.98998998998999	0.229771338901546\\
5.01801801801802	0.232676194703265\\
5.04604604604605	0.235537387886891\\
5.07407407407407	0.238356718946092\\
5.1021021021021	0.241135869383538\\
5.13013013013013	0.24387641238291\\
5.15815815815816	0.246579822284278\\
5.18618618618619	0.249247483022607\\
5.21421421421421	0.25188069566453\\
5.24224224224224	0.254480685158244\\
5.27027027027027	0.2570486063945\\
5.2982982982983	0.259585549662561\\
5.32632632632633	0.262092545573239\\
5.35435435435435	0.264570569511158\\
5.38238238238238	0.267020545670041\\
5.41041041041041	0.269443350717685\\
5.43843843843844	0.271839817131271\\
5.46646646646647	0.274210736238472\\
5.49449449449449	0.276556860995409\\
5.52252252252252	0.278878908528729\\
5.55055055055055	0.281177562465771\\
5.57857857857858	0.283453475073992\\
5.60660660660661	0.285707269228357\\
5.63463463463463	0.287939540223281\\
5.66266266266266	0.290150857443839\\
5.69069069069069	0.292341765909363\\
5.71871871871872	0.294512787701117\\
5.74674674674675	0.29666442328448\\
5.77477477477477	0.298797152735025\\
5.8028028028028	0.300911436876843\\
5.83083083083083	0.303007718340687\\
5.85885885885886	0.305086422548703\\
5.88688688688689	0.30714795863187\\
5.91491491491491	0.30919272028568\\
5.94294294294294	0.311221086569048\\
5.97097097097097	0.313233422650991\\
5.998998998999	0.315230080509168\\
6.02702702702703	0.317211399584027\\
6.05505505505506	0.319177707391948\\
6.08308308308308	0.321129320100469\\
6.11111111111111	0.323066543068429\\
6.13913913913914	0.32498967135359\\
6.16716716716717	0.326898990190108\\
6.1951951951952	0.328794775437998\\
6.22322322322322	0.33067729400658\\
6.25125125125125	0.332546804253713\\
6.27927927927928	0.334403556362495\\
6.30730730730731	0.336247792696947\\
6.33533533533534	0.338079748138107\\
6.36336336336336	0.339899650401824\\
6.39139139139139	0.341707720339467\\
6.41941941941942	0.343504172222637\\
6.44744744744745	0.345289214012926\\
6.47547547547548	0.347063047617659\\
6.5035035035035	0.348825869132504\\
6.53153153153153	0.350577869071752\\
6.55955955955956	0.352319232587044\\
6.58758758758759	0.354050139675223\\
6.61561561561562	0.355770765375969\\
6.64364364364364	0.357481279959839\\
6.67167167167167	0.359181849107243\\
6.6996996996997	0.360872634078921\\
6.72772772772773	0.362553791878371\\
6.75575575575576	0.364225475406716\\
6.78378378378378	0.365887833610421\\
6.81181181181181	0.367541011622256\\
6.83983983983984	0.369185150895894\\
6.86786786786787	0.370820389334465\\
6.8958958958959	0.372446861413424\\
6.92392392392392	0.374064698298011\\
6.95195195195195	0.375674027955609\\
6.97997997997998	0.377274975263251\\
7.00800800800801	0.378867662110551\\
7.03603603603604	0.380452207498269\\
7.06406406406406	0.382028727632753\\
7.09209209209209	0.383597336016461\\
7.12012012012012	0.385158143534753\\
7.14814814814815	0.386711258539149\\
7.17617617617618	0.38825678692722\\
7.2042042042042	0.389794832219281\\
7.23223223223223	0.391325495632037\\
7.26026026026026	0.392848876149329\\
7.28828828828829	0.394365070590122\\
7.31631631631632	0.395874173673861\\
7.34434434434434	0.397376278083319\\
7.37237237237237	0.398871474525053\\
7.4004004004004	0.400359851787582\\
7.42842842842843	0.401841496797387\\
7.45645645645646	0.403316494672827\\
7.48448448448448	0.404784928776081\\
7.51251251251251	0.40624688076318\\
7.54054054054054	0.407702430632241\\
7.56856856856857	0.409151656769952\\
7.5965965965966	0.410594635996413\\
7.62462462462462	0.412031443608379\\
7.65265265265265	0.413462153420995\\
7.68068068068068	0.414886837808068\\
7.70870870870871	0.416305567740952\\
7.73673673673674	0.417718412826098\\
7.76476476476476	0.419125441341322\\
7.79279279279279	0.420526720270849\\
7.82082082082082	0.421922315339185\\
7.84884884884885	0.423312291043851\\
7.87687687687688	0.424696710687045\\
7.9049049049049	0.426075636406256\\
7.93293293293293	0.427449129203887\\
7.96096096096096	0.428817248975916\\
7.98898898898899	0.430180054539633\\
8.01701701701702	0.4315376036605\\
8.04504504504505	0.43288995307815\\
8.07307307307307	0.434237158531578\\
8.1011011011011	0.435579274783534\\
8.12912912912913	0.43691635564417\\
8.15715715715716	0.43824845399395\\
8.18518518518519	0.43957562180586\\
8.21321321321321	0.440897910166944\\
8.24124124124124	0.442215369299191\\
8.26926926926927	0.443528048579791\\
8.2972972972973	0.444835996560787\\
8.32532532532533	0.446139260988152\\
8.35335335335335	0.447437888820293\\
8.38138138138138	0.448731926246025\\
8.40940940940941	0.450021418702015\\
8.43743743743744	0.451306410889725\\
8.46546546546547	0.452586946791872\\
8.49349349349349	0.453863069688413\\
8.52152152152152	0.45513482217208\\
8.54954954954955	0.456402246163482\\
8.57757757757758	0.457665382925771\\
8.60560560560561	0.458924273078919\\
8.63363363363363	0.460178956613581\\
8.66166166166166	0.461429472904596\\
8.68968968968969	0.462675860724099\\
8.71771771771772	0.463918158254299\\
8.74574574574575	0.46515640309989\\
8.77377377377377	0.466390632300148\\
8.8018018018018	0.467620882340692\\
8.82982982982983	0.468847189164943\\
8.85785785785786	0.47006958818527\\
8.88588588588589	0.471288114293857\\
8.91391391391391	0.472502801873275\\
8.94194194194194	0.473713684806787\\
8.96996996996997	0.474920796488384\\
8.997997997998	0.476124169832567\\
9.02602602602603	0.477323837283878\\
9.05405405405405	0.478519830826189\\
9.08208208208208	0.479712181991755\\
9.11011011011011	0.480900921870049\\
9.13813813813814	0.482086081116364\\
9.16616616616617	0.483267689960212\\
9.19419419419419	0.484445778213514\\
9.22222222222222	0.485620375278582\\
9.25025025025025	0.486791510155916\\
9.27827827827828	0.48795921145181\\
9.30630630630631	0.489123507385765\\
9.33433433433433	0.49028442579774\\
9.36236236236236	0.491441994155213\\
9.39039039039039	0.492596239560092\\
9.41841841841842	0.493747188755443\\
9.44644644644645	0.49489486813208\\
9.47447447447447	0.496039303734983\\
9.5025025025025	0.497180521269581\\
9.53053053053053	0.49831854610788\\
9.55855855855856	0.499453403294454\\
9.58658658658659	0.5005851175523\\
9.61461461461461	0.501713713288554\\
9.64264264264264	0.502839214600084\\
9.67067067067067	0.503961645278952\\
9.6986986986987	0.505081028817758\\
9.72672672672673	0.506197388414855\\
9.75475475475475	0.507310746979464\\
9.78278278278278	0.508421127136656\\
9.81081081081081	0.509528551232246\\
9.83883883883884	0.510633041337558\\
9.86686686686687	0.511734619254103\\
9.89489489489489	0.512833306518146\\
9.92292292292292	0.51392912440518\\
9.95095095095095	0.515022093934298\\
9.97897897897898	0.516112235872476\\
10.007007007007	0.517199570738763\\
10.035035035035	0.518284118808384\\
10.0630630630631	0.519365900116751\\
10.0910910910911	0.520444934463399\\
10.1191191191191	0.521521241415828\\
10.1471471471471	0.522594840313277\\
10.1751751751752	0.523665750270412\\
10.2032032032032	0.524733990180936\\
10.2312312312312	0.525799578721137\\
10.2592592592593	0.526862534353349\\
10.2872872872873	0.527922875329351\\
10.3153153153153	0.528980619693694\\
10.3433433433433	0.530035785286965\\
10.3713713713714	0.531088389748981\\
10.3993993993994	0.532138450521917\\
10.4274274274274	0.53318598485338\\
10.4554554554555	0.534231009799418\\
10.4834834834835	0.535273542227462\\
10.5115115115115	0.536313598819228\\
10.5395395395395	0.537351196073541\\
10.5675675675676	0.538386350309121\\
10.5955955955956	0.539419077667306\\
10.6236236236236	0.540449394114727\\
10.6516516516517	0.541477315445924\\
10.6796796796797	0.542502857285926\\
10.7077077077077	0.543526035092764\\
10.7357357357357	0.544546864159951\\
10.7637637637638	0.54556535961891\\
10.7917917917918	0.54658153644135\\
10.8198198198198	0.54759540944161\\
10.8478478478478	0.548606993278948\\
10.8758758758759	0.549616302459793\\
10.9039039039039	0.550623351339957\\
10.9319319319319	0.551628154126796\\
10.95995995996	0.552630724881347\\
10.987987987988	0.553631077520412\\
11.016016016016	0.554629225818611\\
11.044044044044	0.555625183410398\\
11.0720720720721	0.556618963792034\\
11.1001001001001	0.557610580323535\\
11.1281281281281	0.558600046230576\\
11.1561561561562	0.559587374606364\\
11.1841841841842	0.560572578413478\\
11.2122122122122	0.561555670485679\\
11.2402402402402	0.56253666352968\\
11.2682682682683	0.563515570126897\\
11.2962962962963	0.564492402735156\\
11.3243243243243	0.56546717369038\\
11.3523523523524	0.566439895208243\\
11.3803803803804	0.567410579385795\\
11.4084084084084	0.56837923820306\\
11.4364364364364	0.569345883524606\\
11.4644644644645	0.570310527101088\\
11.4924924924925	0.571273180570765\\
11.5205205205205	0.572233855460992\\
11.5485485485485	0.573192563189684\\
11.5765765765766	0.574149315066758\\
11.6046046046046	0.575104122295551\\
11.6326326326326	0.576056995974211\\
11.6606606606607	0.577007947097069\\
11.6886886886887	0.577956986555983\\
11.7167167167167	0.578904125141666\\
11.7447447447447	0.579849373544989\\
11.7727727727728	0.580792742358259\\
11.8008008008008	0.581734242076483\\
11.8288288288288	0.582673883098604\\
11.8568568568569	0.583611675728726\\
11.8848848848849	0.58454763017731\\
11.9129129129129	0.585481756562355\\
11.9409409409409	0.586414064910561\\
11.968968968969	0.587344565158469\\
11.996996996997	0.58827326715359\\
12.025025025025	0.589200180655508\\
12.0530530530531	0.590125315336971\\
12.0810810810811	0.59104868078496\\
12.1091091091091	0.59197028650175\\
12.1371371371371	0.592890141905942\\
12.1651651651652	0.593808256333488\\
12.1931931931932	0.594724639038698\\
12.2212212212212	0.595639299195228\\
12.2492492492493	0.59655224589706\\
12.2772772772773	0.597463488159456\\
12.3053053053053	0.598373034919906\\
12.3333333333333	0.599280895039062\\
12.3613613613614	0.600187077301649\\
12.3893893893894	0.601091590417372\\
12.4174174174174	0.601994443021804\\
12.4454454454454	0.602895643677258\\
12.4734734734735	0.603795200873655\\
12.5015015015015	0.604693123029371\\
12.5295295295295	0.60558941849207\\
12.5575575575576	0.606484095539535\\
12.5855855855856	0.607377162380472\\
12.6136136136136	0.608268627155318\\
12.6416416416416	0.609158497937022\\
12.6696696696697	0.610046782731825\\
12.6976976976977	0.610933489480025\\
12.7257257257257	0.611818626056727\\
12.7537537537538	0.612702200272592\\
12.7817817817818	0.613584219874564\\
12.8098098098098	0.614464692546591\\
12.8378378378378	0.615343625910339\\
12.8658658658659	0.616221027525893\\
12.8938938938939	0.617096904892442\\
12.9219219219219	0.617971265448965\\
12.94994994995	0.618844116574902\\
12.977977977978	0.619715465590811\\
13.006006006006	0.620585319759023\\
13.034034034034	0.621453686284285\\
13.0620620620621	0.622320572314393\\
13.0900900900901	0.623185984940817\\
13.1181181181181	0.624049931199317\\
13.1461461461461	0.624912418070548\\
13.1741741741742	0.625773452480664\\
13.2022022022022	0.626633041301905\\
13.2302302302302	0.62749119135318\\
13.2582582582583	0.628347909400641\\
13.2862862862863	0.629203202158252\\
13.3143143143143	0.630057076288345\\
13.3423423423423	0.63090953840217\\
13.3703703703704	0.631760595060445\\
13.3983983983984	0.632610252773884\\
13.4264264264264	0.633458518003732\\
13.4544544544545	0.634305397162283\\
13.4824824824825	0.635150896613396\\
13.5105105105105	0.635995022673005\\
13.5385385385385	0.636837781609617\\
13.5665665665666	0.637679179644806\\
13.5945945945946	0.638519222953704\\
13.6226226226226	0.63935791766548\\
13.6506506506507	0.640195269863814\\
13.6786786786787	0.641031285587368\\
13.7067067067067	0.641865970830247\\
13.7347347347347	0.642699331542456\\
13.7627627627628	0.643531373630349\\
13.7907907907908	0.644362102957075\\
13.8188188188188	0.645191525343018\\
13.8468468468468	0.646019646566225\\
13.8748748748749	0.64684647236284\\
13.9029029029029	0.647672008427521\\
13.9309309309309	0.648496260413858\\
13.958958958959	0.649319233934786\\
13.986986986987	0.650140934562988\\
14.015015015015	0.650961367831296\\
14.043043043043	0.65178053923309\\
14.0710710710711	0.652598454222683\\
14.0990990990991	0.653415118215714\\
14.1271271271271	0.65423053658952\\
14.1551551551552	0.655044714683519\\
14.1831831831832	0.655857657799579\\
14.2112112112112	0.656669371202384\\
14.2392392392392	0.657479860119797\\
14.2672672672673	0.658289129743215\\
14.2952952952953	0.659097185227926\\
14.3233233233233	0.659904031693457\\
14.3513513513514	0.660709674223914\\
14.3793793793794	0.661514117868328\\
14.4074074074074	0.662317367640988\\
14.4354354354354	0.663119428521773\\
14.4634634634635	0.663920305456481\\
14.4914914914915	0.664720003357152\\
14.5195195195195	0.665518527102386\\
14.5475475475475	0.666315881537665\\
14.5755755755756	0.667112071475657\\
14.6036036036036	0.667907101696533\\
14.6316316316316	0.668700976948264\\
14.6596596596597	0.669493701946927\\
14.6876876876877	0.670285281377002\\
14.7157157157157	0.671075719891664\\
14.7437437437437	0.671865022113075\\
14.7717717717718	0.67265319263267\\
14.7997997997998	0.673440236011442\\
14.8278278278278	0.674226156780221\\
14.8558558558559	0.675010959439952\\
14.8838838838839	0.675794648461966\\
14.9119119119119	0.676577228288257\\
14.9399399399399	0.67735870333174\\
14.967967967968	0.678139077976524\\
14.995995995996	0.678918356578168\\
15.024024024024	0.679696543463941\\
15.0520520520521	0.680473642933075\\
15.0800800800801	0.681249659257023\\
15.1081081081081	0.682024596679701\\
15.1361361361361	0.682798459417737\\
15.1641641641642	0.68357125166072\\
15.1921921921922	0.68434297757143\\
15.2202202202202	0.685113641286088\\
15.2482482482482	0.685883246914583\\
15.2762762762763	0.686651798540708\\
15.3043043043043	0.687419300222389\\
15.3323323323323	0.688185755991915\\
15.3603603603604	0.688951169856158\\
15.3883883883884	0.689715545796803\\
15.4164164164164	0.69047888777056\\
15.4444444444444	0.691241199709385\\
15.4724724724725	0.692002485520698\\
15.5005005005005	0.69276274908759\\
15.5285285285285	0.693521994269037\\
15.5565565565566	0.694280224900107\\
15.5845845845846	0.695037444792165\\
15.6126126126126	0.69579365773308\\
15.6406406406406	0.69654886748742\\
15.6686686686687	0.697303077796656\\
15.6966966966967	0.698056292379359\\
15.7247247247247	0.698808514931389\\
15.7527527527528	0.699559749126095\\
15.7807807807808	0.7003099986145\\
15.8088088088088	0.701059267025492\\
15.8368368368368	0.701807557966012\\
15.8648648648649	0.702554875021233\\
15.8928928928929	0.703301221754746\\
15.9209209209209	0.704046601708743\\
15.9489489489489	0.704791018404189\\
15.976976976977	0.705534475341004\\
16.005005005005	0.706276975998236\\
16.033033033033	0.707018523834234\\
16.0610610610611	0.707759122286818\\
16.0890890890891	0.70849877477345\\
16.1171171171171	0.7092374846914\\
16.1451451451451	0.709975255417911\\
16.1731731731732	0.710712090310368\\
16.2012012012012	0.711447992706452\\
16.2292292292292	0.712182965924308\\
16.2572572572573	0.712917013262701\\
16.2852852852853	0.71365013800117\\
16.3133133133133	0.714382343400191\\
16.3413413413413	0.715113632701323\\
16.3693693693694	0.715844009127366\\
16.3973973973974	0.716573475882509\\
16.4254254254254	0.717302036152482\\
16.4534534534535	0.7180296931047\\
16.4814814814815	0.71875644988841\\
16.5095095095095	0.719482309634841\\
16.5375375375375	0.72020727545734\\
16.5655655655656	0.720931350451516\\
16.5935935935936	0.721654537695384\\
16.6216216216216	0.7223768402495\\
16.6496496496497	0.7230982611571\\
16.6776776776777	0.723818803444236\\
16.7057057057057	0.72453847011991\\
16.7337337337337	0.725257264176208\\
16.7617617617618	0.725975188588434\\
16.7897897897898	0.726692246315238\\
16.8178178178178	0.727408440298745\\
16.8458458458458	0.728123773464687\\
16.8738738738739	0.728838248722526\\
16.9019019019019	0.729551868965583\\
16.9299299299299	0.730264637071159\\
16.957957957958	0.730976555900661\\
16.985985985986	0.731687628299721\\
17.014014014014	0.732397857098321\\
17.042042042042	0.733107245110909\\
17.0700700700701	0.733815795136519\\
17.0980980980981	0.734523509958887\\
17.1261261261261	0.73523039234657\\
17.1541541541542	0.735936445053058\\
17.1821821821822	0.736641670816888\\
17.2102102102102	0.73734607236176\\
17.2382382382382	0.738049652396644\\
17.2662662662663	0.738752413615895\\
17.2942942942943	0.73945435869936\\
17.3223223223223	0.740155490312487\\
17.3503503503504	0.740855811106432\\
17.3783783783784	0.741555323718169\\
17.4064064064064	0.74225403077059\\
17.4344344344344	0.742951934872611\\
17.4624624624625	0.74364903861928\\
17.4904904904905	0.744345344591872\\
17.5185185185185	0.745040855357998\\
17.5465465465465	0.7457355734717\\
17.5745745745746	0.746429501473551\\
17.6026026026026	0.747122641890759\\
17.6306306306306	0.747814997237258\\
17.6586586586587	0.748506570013808\\
17.6866866866867	0.749197362708093\\
17.7147147147147	0.74988737779481\\
17.7427427427427	0.75057661773577\\
17.7707707707708	0.751265084979987\\
17.7987987987988	0.751952781963771\\
17.8268268268268	0.75263971111082\\
17.8548548548549	0.753325874832312\\
17.8828828828829	0.754011275526992\\
17.9109109109109	0.754695915581264\\
17.9389389389389	0.755379797369277\\
17.966966966967	0.756062923253014\\
17.994994994995	0.756745295582378\\
18.023023023023	0.757426916695277\\
18.0510510510511	0.758107788917711\\
18.0790790790791	0.758787914563853\\
18.1071071071071	0.759467295936138\\
18.1351351351351	0.760145935325339\\
18.1631631631632	0.760823835010655\\
18.1911911911912	0.761500997259788\\
18.2192192192192	0.762177424329026\\
18.2472472472472	0.762853118463323\\
18.2752752752753	0.763528081896377\\
18.3033033033033	0.764202316850705\\
18.3313313313313	0.764875825537731\\
18.3593593593594	0.76554861015785\\
18.3873873873874	0.766220672900515\\
18.4154154154154	0.766892015944307\\
18.4434434434434	0.76756264145701\\
18.4714714714715	0.768232551595689\\
18.4994994994995	0.76890174850676\\
18.5275275275275	0.769570234326066\\
18.5555555555556	0.770238011178944\\
18.5835835835836	0.770905081180305\\
18.6116116116116	0.771571446434699\\
18.6396396396396	0.772237109036386\\
18.6676676676677	0.772902071069407\\
18.6956956956957	0.773566334607657\\
18.7237237237237	0.774229901714946\\
18.7517517517518	0.774892774445073\\
18.7797797797798	0.77555495484189\\
18.8078078078078	0.776216444939373\\
18.8358358358358	0.776877246761685\\
18.8638638638639	0.777537362323241\\
18.8918918918919	0.778196793628777\\
18.9199199199199	0.778855542673412\\
18.9479479479479	0.779513611442712\\
18.975975975976	0.780171001912754\\
19.004004004004	0.78082771605019\\
19.032032032032	0.781483755812309\\
19.0600600600601	0.782139123147097\\
19.0880880880881	0.782793819993302\\
19.1161161161161	0.783447848280492\\
19.1441441441441	0.784101209929115\\
19.1721721721722	0.784753906850563\\
19.2002002002002	0.785405940947226\\
19.2282282282282	0.786057314112555\\
19.2562562562563	0.786708028231118\\
19.2842842842843	0.787358085178659\\
19.3123123123123	0.788007486822157\\
19.3403403403403	0.788656235019877\\
19.3683683683684	0.789304331621434\\
19.3963963963964	0.789951778467845\\
19.4244244244244	0.790598577391584\\
19.4524524524525	0.791244730216639\\
19.4804804804805	0.791890238758566\\
19.5085085085085	0.792535104824541\\
19.5365365365365	0.793179330213419\\
19.5645645645646	0.793822916715781\\
19.5925925925926	0.794465866113992\\
19.6206206206206	0.79510818018225\\
19.6486486486486	0.795749860686639\\
19.6766766766767	0.796390909385182\\
19.7047047047047	0.797031328027889\\
19.7327327327327	0.797671118356812\\
19.7607607607608	0.798310282106092\\
19.7887887887888	0.798948821002009\\
19.8168168168168	0.799586736763034\\
19.8448448448448	0.800224031099874\\
19.8728728728729	0.800860705715527\\
19.9009009009009	0.801496762305324\\
19.9289289289289	0.802132202556979\\
19.956956956957	0.802767028150638\\
19.984984984985	0.803401240758927\\
20.013013013013	0.804034842046993\\
20.041041041041	0.804667833672559\\
20.0690690690691	0.805300217285961\\
20.0970970970971	0.805931994530202\\
20.1251251251251	0.806563167040992\\
20.1531531531532	0.807193736446792\\
20.1811811811812	0.807823704368867\\
20.2092092092092	0.808453072421318\\
20.2372372372372	0.809081842211137\\
20.2652652652653	0.809710015338244\\
20.2932932932933	0.810337593395531\\
20.3213213213213	0.810964577968907\\
20.3493493493494	0.811590970637341\\
20.3773773773774	0.8122167729729\\
20.4054054054054	0.812841986540795\\
20.4334334334334	0.813466612899423\\
20.4614614614615	0.814090653600403\\
20.4894894894895	0.814714110188623\\
20.5175175175175	0.815336984202277\\
20.5455455455455	0.815959277172906\\
20.5735735735736	0.81658099062544\\
20.6016016016016	0.817202126078233\\
20.6296296296296	0.817822685043108\\
20.6576576576577	0.818442669025393\\
20.6856856856857	0.819062079523959\\
20.7137137137137	0.819680918031261\\
20.7417417417417	0.820299186033374\\
20.7697697697698	0.820916885010032\\
20.7977977977978	0.821534016434666\\
20.8258258258258	0.822150581774439\\
20.8538538538539	0.822766582490288\\
20.8818818818819	0.823382020036953\\
20.9099099099099	0.823996895863022\\
20.9379379379379	0.82461121141096\\
20.965965965966	0.825224968117152\\
20.993993993994	0.82583816741193\\
21.022022022022	0.826450810719618\\
21.0500500500501	0.827062899458558\\
21.0780780780781	0.827674435041152\\
21.1061061061061	0.828285418873892\\
21.1341341341341	0.828895852357398\\
21.1621621621622	0.82950573688645\\
21.1901901901902	0.830115073850019\\
21.2182182182182	0.830723864631309\\
21.2462462462462	0.831332110607779\\
21.2742742742743	0.831939813151187\\
21.3023023023023	0.832546973627615\\
21.3303303303303	0.833153593397507\\
21.3583583583584	0.833759673815696\\
21.3863863863864	0.834365216231441\\
21.4144144144144	0.834970221988458\\
21.4424424424424	0.835574692424947\\
21.4704704704705	0.83617862887363\\
21.4984984984985	0.83678203266178\\
21.5265265265265	0.837384905111247\\
21.5545545545546	0.837987247538498\\
21.5825825825826	0.838589061254639\\
21.6106106106106	0.83919034756545\\
21.6386386386386	0.839791107771415\\
21.6666666666667	0.840391343167748\\
21.6946946946947	0.840991055044428\\
21.7227227227227	0.841590244686225\\
21.7507507507508	0.84218891337273\\
21.7787787787788	0.842787062378383\\
21.8068068068068	0.843384692972506\\
21.8348348348348	0.843981806419325\\
21.8628628628629	0.844578403978003\\
21.8908908908909	0.84517448690267\\
21.9189189189189	0.845770056442442\\
21.9469469469469	0.846365113841462\\
21.974974974975	0.846959660338915\\
22.003003003003	0.847553697169063\\
22.031031031031	0.84814722556127\\
22.0590590590591	0.84874024674003\\
22.0870870870871	0.849332761924992\\
22.1151151151151	0.849924772330986\\
22.1431431431431	0.850516279168054\\
22.1711711711712	0.851107283641472\\
22.1991991991992	0.851697786951777\\
22.2272272272272	0.852287790294794\\
22.2552552552553	0.85287729486166\\
22.2832832832833	0.853466301838852\\
22.3113113113113	0.854054812408209\\
22.3393393393393	0.85464282774696\\
22.3673673673674	0.855230349027749\\
22.3953953953954	0.855817377418658\\
22.4234234234234	0.856403914083231\\
22.4514514514515	0.856989960180503\\
22.4794794794795	0.85757551686502\\
22.5075075075075	0.858160585286865\\
22.5355355355355	0.858745166591679\\
22.5635635635636	0.859329261920691\\
22.5915915915916	0.859912872410736\\
22.6196196196196	0.86049599919428\\
22.6476476476476	0.861078643399446\\
22.6756756756757	0.861660806150033\\
22.7037037037037	0.862242488565541\\
22.7317317317317	0.862823691761195\\
22.7597597597598	0.863404416847966\\
22.7877877877878	0.863984664932595\\
22.8158158158158	0.864564437117612\\
22.8438438438438	0.865143734501364\\
22.8718718718719	0.865722558178031\\
22.8998998998999	0.866300909237654\\
22.9279279279279	0.866878788766151\\
22.955955955956	0.867456197845341\\
22.983983983984	0.868033137552967\\
23.012012012012	0.868609608962717\\
23.04004004004	0.869185613144243\\
23.0680680680681	0.869761151163183\\
23.0960960960961	0.870336224081183\\
23.1241241241241	0.870910832955918\\
23.1521521521522	0.87148497884111\\
23.1801801801802	0.872058662786552\\
23.2082082082082	0.872631885838126\\
23.2362362362362	0.873204649037823\\
23.2642642642643	0.873776953423767\\
23.2922922922923	0.874348800030228\\
23.3203203203203	0.874920189887649\\
23.3483483483483	0.875491124022661\\
23.3763763763764	0.876061603458106\\
23.4044044044044	0.876631629213052\\
23.4324324324324	0.877201202302817\\
23.4604604604605	0.877770323738986\\
23.4884884884885	0.878338994529429\\
23.5165165165165	0.878907215678323\\
23.5445445445445	0.879474988186168\\
23.5725725725726	0.880042313049806\\
23.6006006006006	0.880609191262443\\
23.6286286286286	0.881175623813662\\
23.6566566566567	0.881741611689445\\
23.6846846846847	0.882307155872191\\
23.7127127127127	0.882872257340733\\
23.7407407407407	0.883436917070356\\
23.7687687687688	0.884001136032818\\
23.7967967967968	0.88456491519636\\
23.8248248248248	0.885128255525734\\
23.8528528528529	0.885691157982213\\
23.8808808808809	0.88625362352361\\
23.9089089089089	0.886815653104296\\
23.9369369369369	0.88737724767522\\
23.964964964965	0.887938408183919\\
23.992992992993	0.888499135574543\\
24.021021021021	0.889059430787867\\
24.049049049049	0.889619294761307\\
24.0770770770771	0.890178728428941\\
24.1051051051051	0.890737732721522\\
24.1331331331331	0.891296308566496\\
24.1611611611612	0.891854456888017\\
24.1891891891892	0.892412178606966\\
24.2172172172172	0.892969474640962\\
24.2452452452452	0.893526345904384\\
24.2732732732733	0.894082793308385\\
24.3013013013013	0.894638817760903\\
24.3293293293293	0.895194420166685\\
24.3573573573574	0.895749601427296\\
24.3853853853854	0.896304362441137\\
24.4134134134134	0.896858704103462\\
24.4414414414414	0.897412627306391\\
24.4694694694695	0.897966132938925\\
24.4974974974975	0.898519221886962\\
24.5255255255255	0.899071895033314\\
24.5535535535536	0.899624153257719\\
24.5815815815816	0.900175997436857\\
24.6096096096096	0.900727428444365\\
24.6376376376376	0.901278447150851\\
24.6656656656657	0.90182905442391\\
24.6936936936937	0.902379251128136\\
24.7217217217217	0.90292903812514\\
24.7497497497498	0.903478416273561\\
24.7777777777778	0.904027386429081\\
24.8058058058058	0.904575949444442\\
24.8338338338338	0.905124106169455\\
24.8618618618619	0.905671857451019\\
24.8898898898899	0.906219204133131\\
24.9179179179179	0.906766147056903\\
24.9459459459459	0.907312687060572\\
24.973973973974	0.907858824979519\\
25.002002002002	0.908404561646277\\
25.03003003003	0.908949897890547\\
25.0580580580581	0.909494834539211\\
25.0860860860861	0.910039372416348\\
25.1141141141141	0.910583512343242\\
25.1421421421421	0.911127255138399\\
25.1701701701702	0.91167060161756\\
25.1981981981982	0.91221355259371\\
25.2262262262262	0.912756108877098\\
25.2542542542543	0.913298271275243\\
25.2822822822823	0.913840040592951\\
25.3103103103103	0.914381417632325\\
25.3383383383383	0.914922403192779\\
25.3663663663664	0.915462998071051\\
25.3943943943944	0.916003203061215\\
25.4224224224224	0.916543018954693\\
25.4504504504505	0.917082446540266\\
25.4784784784785	0.917621486604091\\
25.5065065065065	0.918160139929707\\
25.5345345345345	0.918698407298052\\
25.5625625625626	0.919236289487471\\
25.5905905905906	0.919773787273732\\
25.6186186186186	0.920310901430034\\
25.6466466466466	0.920847632727023\\
25.6746746746747	0.9213839819328\\
25.7027027027027	0.921919949812934\\
25.7307307307307	0.922455537130473\\
25.7587587587588	0.92299074464596\\
25.7867867867868	0.923525573117435\\
25.8148148148148	0.924060023300457\\
25.8428428428428	0.924594095948108\\
25.8708708708709	0.925127791811007\\
25.8988988988989	0.925661111637322\\
25.9269269269269	0.926194056172778\\
25.954954954955	0.926726626160673\\
25.982982982983	0.927258822341883\\
26.011011011011	0.927790645454879\\
26.039039039039	0.928322096235733\\
26.0670670670671	0.928853175418132\\
26.0950950950951	0.929383883733386\\
26.1231231231231	0.929914221910441\\
26.1511511511512	0.930444190675891\\
26.1791791791792	0.930973790753982\\
26.2072072072072	0.931503022866631\\
26.2352352352352	0.932031887733431\\
26.2632632632633	0.932560386071661\\
26.2912912912913	0.933088518596301\\
26.3193193193193	0.933616286020039\\
26.3473473473473	0.934143689053279\\
26.3753753753754	0.934670728404156\\
26.4034034034034	0.935197404778544\\
26.4314314314314	0.935723718880064\\
26.4594594594595	0.936249671410098\\
26.4874874874875	0.936775263067797\\
26.5155155155155	0.937300494550087\\
26.5435435435435	0.937825366551687\\
26.5715715715716	0.938349879765112\\
26.5995995995996	0.938874034880685\\
26.6276276276276	0.939397832586546\\
26.6556556556557	0.939921273568665\\
26.6836836836837	0.940444358510844\\
26.7117117117117	0.940967088094737\\
26.7397397397397	0.941489462999848\\
26.7677677677678	0.94201148390355\\
26.7957957957958	0.942533151481089\\
26.8238238238238	0.943054466405594\\
26.8518518518519	0.943575429348087\\
26.8798798798799	0.944096040977494\\
26.9079079079079	0.944616301960649\\
26.9359359359359	0.945136212962308\\
26.963963963964	0.945655774645157\\
26.991991991992	0.946174987669818\\
27.02002002002	0.946693852694862\\
27.048048048048	0.947212370376815\\
27.0760760760761	0.947730541370168\\
27.1041041041041	0.948248366327386\\
27.1321321321321	0.948765845898916\\
27.1601601601602	0.949282980733196\\
27.1881881881882	0.949799771476663\\
27.2162162162162	0.950316218773763\\
27.2442442442442	0.950832323266958\\
27.2722722722723	0.951348085596737\\
27.3003003003003	0.951863506401621\\
27.3283283283283	0.952378586318172\\
27.3563563563564	0.952893325981005\\
27.3843843843844	0.953407726022792\\
27.4124124124124	0.953921787074273\\
27.4404404404404	0.954435509764262\\
27.4684684684685	0.954948894719657\\
27.4964964964965	0.955461942565447\\
27.5245245245245	0.955974653924722\\
27.5525525525526	0.956487029418677\\
27.5805805805806	0.956999069666625\\
27.6086086086086	0.957510775286002\\
27.6366366366366	0.958022146892373\\
27.6646646646647	0.958533185099446\\
27.6926926926927	0.959043890519074\\
27.7207207207207	0.959554263761265\\
27.7487487487487	0.960064305434189\\
27.7767767767768	0.960574016144189\\
27.8048048048048	0.961083396495782\\
27.8328328328328	0.961592447091674\\
27.8608608608609	0.962101168532763\\
27.8888888888889	0.962609561418147\\
27.9169169169169	0.963117626345133\\
27.9449449449449	0.963625363909243\\
27.972972972973	0.964132774704222\\
28.001001001001	0.964639859322047\\
28.029029029029	0.96514661835293\\
28.0570570570571	0.965653052385331\\
28.0850850850851	0.96615916200596\\
28.1131131131131	0.966664947799788\\
28.1411411411411	0.967170410350051\\
28.1691691691692	0.967675550238262\\
28.1971971971972	0.96818036804421\\
28.2252252252252	0.968684864345977\\
28.2532532532533	0.969189039719937\\
28.2812812812813	0.969692894740768\\
28.3093093093093	0.970196429981455\\
28.3373373373373	0.970699646013299\\
28.3653653653654	0.971202543405926\\
28.3933933933934	0.97170512272729\\
28.4214214214214	0.972207384543682\\
28.4494494494495	0.972709329419735\\
28.4774774774775	0.973210957918433\\
28.5055055055055	0.973712270601118\\
28.5335335335335	0.974213268027492\\
28.5615615615616	0.974713950755629\\
28.5895895895896	0.975214319341981\\
28.6176176176176	0.975714374341381\\
28.6456456456456	0.976214116307051\\
28.6736736736737	0.976713545790611\\
28.7017017017017	0.977212663342084\\
28.7297297297297	0.977711469509901\\
28.7577577577578	0.978209964840907\\
28.7857857857858	0.978708149880372\\
28.8138138138138	0.979206025171993\\
28.8418418418418	0.979703591257899\\
28.8698698698699	0.980200848678664\\
28.8978978978979	0.980697797973305\\
28.9259259259259	0.981194439679295\\
28.953953953954	0.981690774332566\\
28.981981981982	0.982186802467513\\
29.01001001001	0.982682524617006\\
29.038038038038	0.98317794131239\\
29.0660660660661	0.983673053083496\\
29.0940940940941	0.984167860458642\\
29.1221221221221	0.984662363964645\\
29.1501501501502	0.985156564126821\\
29.1781781781782	0.985650461468995\\
29.2062062062062	0.986144056513505\\
29.2342342342342	0.986637349781208\\
29.2622622622623	0.987130341791488\\
29.2902902902903	0.987623033062258\\
29.3183183183183	0.988115424109968\\
29.3463463463463	0.988607515449612\\
29.3743743743744	0.989099307594731\\
29.4024024024024	0.98959080105742\\
29.4304304304304	0.990081996348334\\
29.4584584584585	0.990572893976694\\
29.4864864864865	0.991063494450291\\
29.5145145145145	0.991553798275491\\
29.5425425425425	0.992043805957245\\
29.5705705705706	0.992533517999089\\
29.5985985985986	0.993022934903154\\
29.6266266266266	0.993512057170167\\
29.6546546546547	0.994000885299461\\
29.6826826826827	0.994489419788979\\
29.7107107107107	0.994977661135276\\
29.7387387387387	0.995465609833528\\
29.7667667667668	0.995953266377539\\
29.7947947947948	0.996440631259739\\
29.8228228228228	0.996927704971198\\
29.8508508508509	0.997414488001626\\
29.8788788788789	0.997900980839379\\
29.9069069069069	0.998387183971464\\
29.9349349349349	0.998873097883548\\
29.962962962963	0.999358723059955\\
29.990990990991	0.999844059983681\\
30.019019019019	1\\
30.047047047047	1\\
30.0750750750751	1\\
30.1031031031031	1\\
30.1311311311311	1\\
30.1591591591592	1\\
30.1871871871872	1\\
30.2152152152152	1\\
30.2432432432432	1\\
30.2712712712713	1\\
30.2992992992993	1\\
30.3273273273273	1\\
30.3553553553554	1\\
30.3833833833834	1\\
30.4114114114114	1\\
30.4394394394394	1\\
30.4674674674675	1\\
30.4954954954955	1\\
30.5235235235235	1\\
30.5515515515516	1\\
30.5795795795796	1\\
30.6076076076076	1\\
30.6356356356356	1\\
30.6636636636637	1\\
30.6916916916917	1\\
30.7197197197197	1\\
30.7477477477477	1\\
30.7757757757758	1\\
30.8038038038038	1\\
30.8318318318318	1\\
30.8598598598599	1\\
30.8878878878879	1\\
30.9159159159159	1\\
30.9439439439439	1\\
30.971971971972	1\\
31	1\\
};
\addlegendentry{Almost Square Root};

\end{axis}
\end{tikzpicture}%
	\caption{The different color translation functions.}
	\label{fig:colors}
\end{figure}

The smiley is draws using some circles and a quadratic curve. The code is visible in Listing \ref{lst:smiley.js}.


\includecode[js]{smiley.js}{resources/smiley.js}{lst:smiley.js}

\end{document}