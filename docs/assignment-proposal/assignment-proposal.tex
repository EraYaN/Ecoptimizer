% !TeX spellcheck = en_GB
% !TEX program = xelatex+makeindex+bibtex
\documentclass[final,a4paper,11pt]{article}
\input{../library/preamble.tex}
\input{../library/style.tex}
\addbibresource{../library/bibliography.bib}
\author{Erwin de Haan}
\title{Assignment Proposal}
\date{\today}
\begin{document}
\maketitle

\section{Ecoptimizer}
The Ecoptimizer is the part shown after the customer has entered their details.
It gives the customer a choice between swift delivery, effort and eco-impact.
Fairphone has hinted on that their system will already have a ``eco-impact counter''.
This will integrate into that system.

The main goal on the module it for the user to choose their wanted method of delivery and force the user to think about the impact that delivery method has.

For example the options a user could choose from are TD's, pickup points like stores and warehouses, parcel delivery to their home etcetera.

As for the impact calculation, the part would get extra ``impact'' for every step in the delivery process on top of a baseline ``manufacturer-to-dutch-warehouse'' impact.
These impacts could be calculated in the same way transport would be calculated for an LCA.

I think Fairphone has chosen the units ``grams of $\textrm{CO}_\textrm{2}$''.
I will verify this if needed.
Other options include all kinds of point systems used throughout the minor's courses.

I think it is a good that I will email big progresses so it can get a quick look and see if I'm going in the right direction.

\subsection{Planning}
This project will start at the start on the last period (20th of April).
I think I need 4 to 5 weeks to finish the prototype and the first batch of prototyping.
Then an other week to tweak the prototype.

\subsection{Deliverables}
\begin{itemize}
	\item Prototype usable for user testing
	\item User testing results and actions taken to implement these results
	\item Final prototype
	\item Documentation on the mathematical model used to calculate the eco-impacts
\end{itemize}


\end{document}